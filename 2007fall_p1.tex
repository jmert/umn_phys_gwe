%%%%%%%%%%%%%%%%%%%%%%%%%%%%%%%%%%%%%%%%%%%%%%%%%%%%%%%%%%%%%%%%%%%%%%%%%%%%%%%
%%%% Problem 9
%%%%%%%%%%%%%%%%%%%%%%%%%%%%%%%%%%%%%%%%%%%%%%%%%%%%%%%%%%%%%%%%%%%%%%%%%%%%%%%
\problem{9}
\subsubsection{Question}
% Keywords
	\index{thermodynamics!Light bulb as blackbody radiator}
	\index{circuits!Light bulb as blackbody radiator}

An electric bulb is rated at \SI{100}{\W} when used with a DC voltage of \SI
{110}{\V}. What total power is dissipated if this voltage is applied to two
such bulbs connected in series? It can be assumed that each bulb dissipates
heat by radiation from its filament similar to a black body and that the
resistance of the filament is proportional to its absolute temperature.

\subsubsection{Answer}

Beginning from the known properties of a single bulb, we know that the
dissipated power in a single bulb $P₀$ is like a black body, so the power
must follow the Stefan-Boltmann law:
\begin{align*}
	P₀ &= σ_B {T₀}⁴
\end{align*}
where $T₀$ is operating equilibrium temperature. In addition, we are told
that the bulb is like a resistor with a resistance proportional to its
temperature:
\begin{align*}
	P₀ &= \frac{V²}{R} = \frac{V₀²}{C T₀}
\end{align*}
Combining the two equations gives the proportionality constant for a single
bulb in the circuit.
\begin{align*}
	C &= \frac{{V₀}²}{σ_B {T₀}⁵}
\end{align*}

When a second bulb is added to the circuit, the voltage across each bulb is
dropped and a corresponding change in the equilibrium temperature is
created. Since the bulbs are in series, the voltage $V = \frac 12 V₀$ across
each resistor sums in series. Repeating the same procedure as in the first
case,
\begin{align*}
	σ_B T⁴ &= 2 \frac{V²}{R} = 2 \frac{(\frac 12 V₀)²}{C T} \\
	σ_B T⁵ &= \frac 12 \frac{{V₀}²}{C}
\end{align*}
and inserting the constant $C$,
\begin{align*}
	T⁵ &= \frac 18 {T₀}⁵ \\
	T &= \frac{1}{2^{1/5}} T₀
\end{align*}
Therefore from the Stefan-Boltmann law, the total power dissipated is
\begin{align*}
	P &= σ_B T⁴ = \frac{1}{2^{4/5}} σ_B {T₀}⁴
\end{align*}
\begin{align}
	\boxed{ P = \frac{\SI{100}{\W}}{2^{4/5}} = \SI{57.435}{\W} }
\end{align}

