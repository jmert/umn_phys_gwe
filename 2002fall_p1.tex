%%%%%%%%%%%%%%%%%%%%%%%%%%%%%%%%%%%%%%%%%%%%%%%%%%%%%%%%%%%%%%%%%%%%%%%%%%%%%%%
%%%% Problem 5
%%%%%%%%%%%%%%%%%%%%%%%%%%%%%%%%%%%%%%%%%%%%%%%%%%%%%%%%%%%%%%%%%%%%%%%%%%%%%%%
\problem{5}
\subsubsection{Question}
% Keywords
	\index{mechanics!Elastic collision on spring-connected blocks}
	\index{Lagrangian!Elastic collision on spring-connected blocks}

Blocks of mass $m$ and $2m$ are free to slide without friction on a
horizontal wire. They are connected by a massless spring of equilibrium
length $L$ and force constant $k$. A projectile of mass $m$ is fired with
velocity $v$ into the block with mass $m$ and sticks to it. If the blocks
are initially at rest, what is the maximum displacement between them in the
subsequent motion?

\subsubsection{Answer}

Take time $t=0$ to be the moment the projectile collides with the mass $m$,
and let the subsequent transfer of momentum be instantaneous. In this case,
the initial conditions of the problem are then:
\begin{align*}
    x₁(0) &= 0			& \dot x₁(0) &= u \\
    x₂(0) &= L			& \dot x₂(0) &= 0
\end{align*}
where $u$ is the initial velocity of the combined project-mass system. We get
$u$ from conservation of mometum:
\begin{align*}
    2mu &= mv + 0 \\
    u &= \frac 12 v
\end{align*}

Now solve the mechanics problem using the Lagrangian approach. Both masses have
kinetic energy, and the spring stores potential energy, so
\begin{align*}
    T &= m{\dot x₁}² + m{\dot x₂}² \\
    V &= \frac 12 m (x₂ - x₁)² \\
    L &= m ({\dot x₁}² + {\dot x₂}²) - \frac 12 k({x₁}² + {x₂}² + 2x₁x₂)
\end{align*}
Setting up the differential equation, we get
\begin{align*}
    \frac{∂L}{∂x₁} &= -kx₁ + kx₂	& \frac{d}{dt}
	\left[\frac{∂L}{∂\dot x₁}\right] &= 2m \ddot x₁ \\
    \frac{∂L}{∂x₂} &=  kx₁ - kx₂	& \frac{d}{dt}
	\left[\frac{∂L}{∂\dot x₁}\right] &= 2m \ddot x₂ \\
\end{align*}
Leading to the system of equations where $ω² = k/2m$,
\begin{align*}
    \begin{bmatrix} \ddot x₁ \\ \ddot x₂ \end{bmatrix} &=
	\begin{bmatrix} -ω² & ω² \\ ω² & -ω² \end{bmatrix}
	\begin{bmatrix} x₁ \\ x₂ \end{bmatrix}
\end{align*}
Solving the eigensystem, we find the eigenfrequencies to be $λ = \{0, -2ω²\}$.
Letting ${ω'}² = 2ω²$, the eigenfunction equations are then
\begin{align*}
    \ddot ψ₁ &= 0
	& \rightarrow&&
	ψ₁ &= A₁t + B₁ \\
    \ddot ψ₂ &= -2ω² ψ₂
	& \rightarrow&&
	ψ₂ &= A₂\cos(ω't) + B₂\sin(ω't)
\end{align*}
From the eigenvectors, we express the solutions of $x₁$ and $x₂$ in terms of
$ψ₁$ and $ψ₂$:
\begin{align*}
    \begin{bmatrix} x₁ \\ x₂ \end{bmatrix} &=
	\begin{bmatrix} 1 & 1 \\ 1 & -1 \end{bmatrix}
	\begin{bmatrix} ψ₁ \\ ψ₂ \end{bmatrix}
\end{align*}
\begin{align*}
    x₁ &= A₁t + B₁ + A₂\cos(ω't) + B₂\sin(ω't) \\
    x₂ &= A₁t + B₁ - A₂\cos(ω't) - B₂\sin(ω't)
\end{align*}
Applying the boundary conditions, we find that
\begin{align*}
    x₁(t) &= \frac 14 vt + \frac 12 L - \frac 12 L\cos(ω't) +
	\frac{v}{4ω'}\sin(ω't) \\
    x₂(t) &= \frac 14 vt + \frac 12 L + \frac 12 L\cos(ω't) -
	\frac{v}{4ω'}\sin(ω't)
\end{align*}
The distance $ℓ(t) = x₂(t) - x₁(t)$ between the two masses maximizes when
\begin{align*}
    \frac{dℓ}{dt} = 0 &= \frac{d}{dt} \left[ L\cos(ω't) -
	\frac{v}{2ω'}\sin(ω't) \right] \\
    t &= -\frac{1}{ω'} \arctan (\frac{v}{2Lω'})
\end{align*}
Plugging back into the function $ℓ(t)$,
\begin{align*}
    ℓ &= L\cos \left[ -\arctan (\frac{v}{2Lω'}) \right] - \frac{v}{2ω'}
	\sin \left[ -\arctan (\frac{v}{2Lω'}) \right] \\
    ℓ &= L \frac{2Lω'}{\sqrt{v² + 4L² {ω'}²}} + \frac{v}{2ω'}
	\frac{v}{\sqrt{v² + 4L² {ω'}²}} \\
    ℓ &= \frac{\sqrt{v² + 4L² {ω'}²}}{2ω'}
\end{align*}
Finally, substituting back in $ω' = \sqrt{2k/m}$ and simplifying, we get the
final solution that maximum distance between the two masses is
\begin{empheq}[box=\fbox]{align}
    ℓ &= \sqrt{L² + \frac{\frac 12 mv²}{8k}}
\end{empheq}
which agrees qualitatively with the fact that a larger spring constant should
stiffen the system and decrease the maximum displacement, while launching the
projectile with a greater velocity would increase it.

%%%%%%%%%%%%%%%%%%%%%%%%%%%%%%%%%%%%%%%%%%%%%%%%%%%%%%%%%%%%%%%%%%%%%%%%%%%%%%%
%%%% Problem 10
%%%%%%%%%%%%%%%%%%%%%%%%%%%%%%%%%%%%%%%%%%%%%%%%%%%%%%%%%%%%%%%%%%%%%%%%%%%%%%%
\problem{10}
\subsubsection{Question}
% Keywords
	\index{thermodynamics!Freezing ice}

Ice on a pond is \SI{10}{\cm} thick and the water temperature just below the
ice is \SI{0}{\celsius}. If the air temperature is \SI{-20}{\celsius}, by
how much will the ice thickness increase in 1 hour? Assuming that the air
temperature stays the same over a long period, how will the ice thickness
increase with time? Comment on any approximation that you make in your
calculation.

Density of ice ${}= \SI{0.9}{\g\per\cm\cubed}$

Thermal conductivity of ice ${}= \SI{0.0005}{\cal\per\cm\per\s\per\celsius}$

Latent heat of fusion of water ${}= \SI{80}{\cal\per\g}$

\subsubsection{Answer}

Since the thermal heat flow is a one dimensional problem, immediately
consider everything with respect to a small area element with its normal
perpendicular to the ice-water interface $dA$. Then we want to know how much
ice is generated on the surface of the ice. This small ice element's mass is
simply
\begin{align*}
    dm &= ρ\,dA\,dz
\end{align*}
where $dz$ is the thickness of the new ice layer. To generate this ice, the
latent heat of fusion must be conducted away, so the energy released is,
\begin{align*}
    dE_f &= L_f\,dm \\
    {} &= L_f ρ \,dA\,dz
\end{align*}

The energy flow is through the ice, and we expect this to increase with the
temperature differential across the ice sheet, suggesting that the thermal
conductivity $κ$ be multiplied by the temperature difference $ΔT$. Furthermore,
the ice will decrease the rate of heat flow as it becomes thicker, so the
quantity should also be divided by the thickness $z$. This gives
\begin{align*}
    \frac{κ ΔT}{z} &= \left[ \si{\cal\per\cm\squared\per\s}
	\right]
\end{align*}
This energy is flowing through a surface element $dA$, giving the power flow
due to heat as
\begin{align*}
    \frac{κΔT\,dA}{z} &= \left[ \si{\cal\per\s} \right]
\end{align*}

This power can be matched in units with the energy released from the ice
calculated above by taking the time derivative of $dE_f$, so equating the two
we have
\begin{align*}
    L_f ρ \,dA\frac{dz}{dt} &= \frac{κΔT\,dA}{z} \\
    ∫_{z₀}^{z₀+δz} z\,dz &= ∫_0^t \frac{κΔT}{L_f ρ}\,dt \\
    2z₀ δz + (δz)² &= \frac{κΔT}{L_f ρ}t
\end{align*}
Solving for the length the ice grows $δz$,
\begin{align*}
    δz &= \frac{-2z₀ ± \sqrt{4{z₀}² - 4(\frac{κΔT}{L_f ρ})t} }{2} \\
    δz &= z₀(1 ± \sqrt{1 - \frac{κΔT}{L_f ρ {z₀}²} t})
\end{align*}
The two roots give solutions $δz = \{ \SI{0.0501}{\cm}, \SI{19.950}{\cm} \}$.
Since the second root is unrealistic, we know that the solution must then be
\begin{empheq}[box=\fbox]{align}
	δz &= \SI{0.0501}{\cm} \quad\text{in an hour}
\end{empheq}

