%%
% Package which provides some minimal common elements for use by all defaults-*
% files.
%
% See defaults.tex for more info.
%
% AUTHORS:      Justin Willmert ‹justin@jdjlab.com›
% SOURCES:      http://www.jdjlab.com/hg/latex/
%
% CONTRIBUTORS:
%   1. Ian-Mathew Hornburg ‹imhornburg@gmail.com›
%      (none — by personal correspondence)
%
%%

%% Load several commonly-assumed commands or macros to be defined
    \RequirePackage{defaults-utilities}

%% Add fancy font features if using an advanced TeX engine.
\ifxeluatex\usepackage{fontspec}

    % Set font features which should be true for all fonts loaded by default
    \defaultfontfeatures{
        % Permit using TeX-style ligatures (i.e. --- for em-dash)
        Ligatures = TeX
    }

    % Also allow math to input using Unicode by default
    \usepackage{unicode-math}

    % Skip resetting the fonts if we're in draft mode
    \ifdraft{\relax}{
        % Use the XITS Math font by default
        \setmathfont[
                % Use italic fonts at all times
                math-style = ISO,
                bold-style = ISO,
                % Force \epsilon,\varepsilon and \phi,\varphi match the unicode
                % character ε,ϵ and φ,ϕ
                vargreek-shape = unicode
            ]{XITS Math}
        % Additionally, use an alternate stylistic set for caligraphic letters
        % so that they are different from the script letters
        \setmathfont[range={\mathcal,\mathbfcal},StylisticSet=1]{XITS Math}
        % Finally, request that certain symbols be provided by Latin Modern
        % Math instead of XITS Math since I think the LM ones looks much nicer.
        \setmathfont[range={"220F}]{Latin Modern Math}  % Product
        \setmathfont[range={"2210}]{Latin Modern Math}  % Coproduct
        \setmathfont[range={"2211}]{Latin Modern Math}  % Summation
        % Use the Asana Math for \top (Unicode Down Tack) since it's slightly
        % better when used as a transpose symbol.
        \setmathfont[range=\top]{Asana Math}    % Down Tack

        % Have the Linux Libertine family of fonts provide the default serif
        % and sans-serif fonts
        \setmainfont{Linux Libertine O}
        \setsansfont{Linux Biolinum O}
    } %\ifdraft

\else %\ifxeluatex

    % Allow pdfLaTeX to parse Unicode without choking. Note, though, that a
    % document *probably* still won't compile since the mapping from character
    % code to font glyph isn't setup by this. Additional work is needed yet
    % before this can happen automatically (if ever).
    %
    % Currently, my use is to allow \ifxeluatex ... \else ... \fi guards to be
    % used around tricky spots and not have pdflatex completely barf as it
    % parses those.
    \usepackage[T1]{fontenc}
    \usepackage[utf8]{inputenc}

    \ifdraft{\relax}{
        % Also use Linux Libertine fonts in pdflatex mode, but do this through
        % the libertine package.
        \usepackage{libertine}

        % Also choose to use use the STIX fonts like the XeLaTeX version
        \usepackage[notext]{stix}

        % Inconsolata is sized much more appropriately for the other chosen
        % fonts
        \usepackage{inconsolata}
    }

\fi %\ifxeluatex
