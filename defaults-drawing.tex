%%
% Collection of LaTeX input files which provide many convenient changes to the
% default environment.
%
% See defaults.tex for more info.
%
% AUTHORS:      Justin Willmert ‹justin@jdjlab.com›
% SOURCES:      http://www.jdjlab.com/hg/latex/
%
% CONTRIBUTORS:
%   1. Ian-Mathew Hornburg ‹imhornburg@gmail.com›
%      (none — by personal correspondence)
%
%==============================================================================
%
% ********************
% CONFIGURABLE OPTIONS
% ********************
%
% 1. To enable the tightpages environment, add the line
%        \PassOptionsToPackage{active}{preview}
%    before loading this file.
%
%%

%% Provide a package-like environment where internal macros are accessible
\makeatletter

%% Load TikZ and pgfplots libraries
\usepackage{pgfplots}
	\usetikzlibrary{calc}
	% Use the command \tikzexternalize to enable externalization
	\usetikzlibrary{external}
	% Then set the external call to match the currently running invocation of
	% TeX. Note that the extra spaces are critical since LaTeX gobbles the
	% separating space between \externlatex and \tikzexternalcheckshellescape,
	% so we include it in the command so that the system call works.
	\ifxetex
		\def\externlatex{xelatex }
	\else
		\ifluatex
			\def\externlatex{lualatex }
		\else
			\def\externlatex{pdflatex }
		\fi
	\fi
	% Now set the external command to use the appropriate LaTeX engine
	\tikzset{external/system call={\externlatex \tikzexternalcheckshellescape
		-halt-on-error -interaction=batchmode -jobname "\image" "\texsource"}}

	% It is assumed that defaults.tex is loaded before this file, so also
	% patch tikzpicture environments to respect the change in font colors
	%
	% Taken from I.-M.:
	\AtBeginEnvironment{tikzpicture}{\color{textcolor}}


%% Enable use of the preview package to isolate environments as PDF pages
%\usepackage[
%		tightpages
%	]{preview}

%% Hide internal macros again
\makeatletter
