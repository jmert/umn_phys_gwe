%%%%%%%%%%%%%%%%%%%%%%%%%%%%%%%%%%%%%%%%%%%%%%%%%%%%%%%%%%%%%%%%%%%%%%%%%%%%%%%
%%%% Problem 1
%%%%%%%%%%%%%%%%%%%%%%%%%%%%%%%%%%%%%%%%%%%%%%%%%%%%%%%%%%%%%%%%%%%%%%%%%%%%%%%
\problem{1}
\subsubsection{Question}
% Keywords
	\index{mechanics!Bead on a hoop}

A particle of mass $m$ is constrained to move without friction on a circular
wire of radius $R$ rotating with constant angular frequency $ω$ about a
vertical diameter. Gravity can not be neglected.
\begin{enumerate}[a)]
    \item
        Write down the Lagrangian for the system and the equations of motion.
    \item
        Find the equilibrium position(s) of the particle and determine
        whether this position is stable.
    \item
        Calculate the frequency of small oscillations about any stable points.
\end{enumerate}

\begin{figure}[H]
    \centering
    \begin{tikzpicture}
        % The axis
        \draw [dashed,->] (0,-1) -- (0,4);
        % The hoop
        \draw (0,1.5) circle (1.5);
        % The bead
        \draw ($(0,1.5)!1.5cm!30:(0,0)$)
            node[circle,fill=black,anchor=center,inner sep=2pt] {}
            coordinate (bead);
        % Label the angle
        \draw [dashed] (0,1.5) -- (bead);
        \draw ($(0,1.5)!0.5cm!(0,0)$)
            coordinate (arc start)
            arc (-90:-60:0.5cm)
            coordinate (arc end);
        \draw ($(arc start)!0.5!(arc end)$)
            node [anchor=north] {$\,\,θ$};
    \end{tikzpicture}
\end{figure}

\subsubsection{Answer}

To start constructing the Lagrangian and equations of motion, we first specify
the kinetic and potential energies. For the kinetic energy, there is an energy
associated with the rotation about the axis and one along the bead. These
combined to give
\begin{align*}
    T &= \frac 12 m (Rω\sin θ)² + \frac 12 m (R\dot θ)² \\
    {} &= \frac 12 m R² ω² \sin² θ + \frac 12 m R² {\dot θ}²
\end{align*}
The potential energy is all gravitational, so
\begin{align*}
    V &= -mgR\cos θ
\end{align*}
where the zero point was taken to be at the center of the hoop to avoid adding
extra constant terms to the Lagrangian. Combining the two, we get
\begin{align}
    \boxed{
    \mathcal L = \frac 12 mR²ω²\sin² θ + \frac 12 mR²{\dot θ}² + mgR\cos θ
    }
\end{align}
Taking the appropriate derivatives in $θ$, the equation of motion is
\begin{align}
    \boxed{
    \ddot θ = ω² \sin θ \cos θ - \frac{g}{R}\sin θ
    }
\end{align}

In order to determine any possible stable points, we note that a stable point
is a place where the angle does not change in time. Since this also equates to
$\ddot θ = 0$, we set the equation above equal to zero and solve for the angles
which satisfy this condition. They end up being the trivial $θ = \{0, π\}$
where the sine function is zero as well as
\begin{align*}
    \cos θ₀ &= \frac{g}{Rω²}
\end{align*}
The three stable points are then
\begin{align}
    \boxed{ θ₀ = \left\{ 0, \arccos(\frac{g}{Rω²}), π \right\} }
\end{align}

To determine the stability of each, we must determine whether we get
oscillatory or exponential solutions to the differential equation of motion.
To do this, we suppose the angle $θ$ is composed of the equilibrium angle
$θ₀$ and a small perturbation $δ$. Expanding the equation in terms of this,
\begin{align*}
    \ddot δ = ω²\sin(θ₀+δ)\cos(θ₀+δ) - \frac{g}{R}\sin(θ₀+δ)
\end{align*}
Using several trigonometric expansions, the equation can be expanded into the
form
\begin{align*}
    \ddot δ = ω²\left[ \cos θ₀\sin θ₀ (\cos²δ - \sin²δ) + \cos δ\sin δ
        (\cos²θ₀ - \sin²θ₀) \right] - \frac{g}{R}\left[ \sin θ₀\cos δ +
        \cos θ₀\sin δ \right]   
\end{align*}

For the case where $θ₀ = 0$,
\begin{align*}
    \ddot δ &= ω² \cos δ \sin δ - \frac{g}{R} \sin δ
\intertext{Expanding to first order in $δ ≈ 0$,}
    \ddot δ &= -(\frac{g}{R} - ω²)δ
\end{align*}
Therefore, the equilibrium point $θ₀ = 0$ is only stable if $ω <
\sqrt{\frac{g}{R}}$.

Likewise for for $θ₀ = π$,
\begin{align*}
    \ddot δ &= (\frac{g}{R} + ω²) δ
\end{align*}
The coefficient on $δ$ will never be negative, so the angle $θ₀ = π$ will
be unstable under all conditions.

For the final angle where $θ₀ = \arccos(\frac{g}{Rω²})$, we must do several
substitutions and expansions. $\cos θ₀$ is trivial. $\sin θ₀$ ends up being
$\sqrt{Rω² - g²}/(Rω²)$ by triangle relations. If we substitute these in plus
do an expansion to first order for small $δ$, we get the equation
\begin{align*}
    \ddot δ &= ω² \left[ \frac{g\sqrt{Rω²-g²}}{R²ω²} + δ \frac{2g²-Rω²}{R²ω²}
        \right] - \frac{g}{R} \left[\frac{\sqrt{Rω²-g²}}{R²ω²} + δ
        \frac{g}{Rω²} \right]
\end{align*}
If we consider only the homogeneous terms dependent on $δ$,
\begin{align*}
    \ddot δ &= \frac{g² - Rω²}{R²ω²} δ
\end{align*}
This equation is stable if and only if the coefficient on $δ$ is negative, so
it must be that $ω > \frac{g}{\sqrt{R}}$.

In summary, the equilibrium points have the following conditions:
\begin{align}
    \boxed{ θ₀ = 0 \quad\quad \text{Stable iff } ω < \sqrt{\frac gR} }
\end{align}
\begin{align}
    \boxed{ θ₀ = \arccos(\frac{g}{Rω²}) \quad\quad \text{Stable iff }
        ω > \frac{g}{\sqrt{R}} }
\end{align}
\begin{align}
    \boxed{ θ₀ = π  \quad\quad \text{Never stable} }
\end{align}


About the two stable points, we simply use the coefficient that has already
been isolated to determine the frequency of the oscillations about that point.
\begin{align}
    \boxed{ ω₁ = \sqrt{\frac{g}{R} - ω²} \quad\quad\text{for $θ₀ = 0$} }
\end{align}
\begin{align}
    \boxed{ ω₂ = \sqrt{\frac{Rω² - g²}{R²ω²}}\quad\quad\text{for $θ₀ =
        \arccos(\frac{g}{Rω²})$} }
\end{align}

%%%%%%%%%%%%%%%%%%%%%%%%%%%%%%%%%%%%%%%%%%%%%%%%%%%%%%%%%%%%%%%%%%%%%%%%%%%%%%%
%%%% Problem 2
%%%%%%%%%%%%%%%%%%%%%%%%%%%%%%%%%%%%%%%%%%%%%%%%%%%%%%%%%%%%%%%%%%%%%%%%%%%%%%%
\problem{2}
\subsubsection{Question}
% Keywords
	\index{electrostatics!Charged sphere in uniform electric field}
    \index{Gauss' Law!Charged sphere in uniform electric field}

The general solution of the Laplace's equation for an electrostatic problem
having azimuthal symmetry can be written as
\begin{align*}
    V(r,θ) &= \sum_{ℓ=0}^∞ (A_ℓ r^ℓ + \frac{B_ℓ}{r^{ℓ+1}}) P_ℓ(\cos θ)
\end{align*}
Now consider the follwoing problem. A solid spherical conductor of radius $R$
having charge $Q$ is placed in an otherwise uniform electric field $\vec E
= E₀\hat z$.
\begin{enumerate}[(a)]
    \item
        Qualtitatively describe the electric field inside and outside of the
        sphere.
    \item
        Solve the problem and find the electric potential in th eregion outside
        the sphere.
\end{enumerate}

\subsubsection{Answer}

To provide a qualitative description, we make use of several properties of
conductors. The electric field inside the conductor is guaranteed to be zero
in the limit of a perfect conductor which can move its electrons anywhere
they're needed to cancel any applied fields. For the region outside the
sphere, it is easiest to describe the region just outside the surface and
the region at infinitely large distances. Far away, the effects of the the
sphere are negligible and the electric field is the external uniform field.
Near the surface, though, all field lines are perpendicular to the surface;
therefore, the external field's lines are curved so that any intersections
occur perpendicular to the surface.

In summary
\par\fbox{\begin{minipage}{\textwidth}
    \begin{enumerate}
        \item
            The field is uniform at large distances
        \item
            The field is perpendicular to the surface of the conductor at
            the conductor's surface
        \item
            There is no field within the interior of the conductor
    \end{enumerate}
\end{minipage}}
\vspace{\baselineskip}

In order to analyze the problem analytically, we make use of the
superposition principle to simplify the problem. Because the sphere carries
its own charge, we treat this case as a superposition of the two simpler
cases of a charged sphere in vacuum and that of a perfectly conducting,
grounded sphere in a uniform electric field.

Since we are only concerned with the potential outside the sphere, we can use
Gauss' Law to get the potential due to the charge $Q$. It is
\begin{align*}
    V_Q(r,θ) &= \frac{Q}{4πε₀r} & r &> R
\end{align*}

The uniform field is the considered by satisfying the appropriate boundary
conditions to solve for the coefficients $A_ℓ$ and $B_ℓ$ in the general
solution given above. We start by converting the given electric field to a
potential. In Cartesian coordinates,
\begin{align*}
    \vec E = E₀\hat z = -\vec ∇V_∞ \quad\quad\Rightarrow\quad\quad V_∞ &= -E₀z
\end{align*}
which when converted to spherical coordinates gives the potential as $r→∞$
\begin{align*}
    V_∞ &= -E₀r\cos θ
\end{align*}
In the infinite distance limit, $B_ℓ/r^{ℓ+1} → 0$ so the boundary condition
equation becomes
\begin{align*}
    -E₀r\cos θ &= \sum_{ℓ=0}^∞ A_ℓ r^ℓ P_ℓ(\cos θ) \\
\intertext{By the orthogonality of the Legendre polynomials, only $A₁$ is
non-zero:}
    -E₀r\cos θ &= A₁r\cos θ \\
    A₁ &= -E₀
\end{align*}

The general solution has thus been simplified to
\begin{align*}
    V₀(r,θ) &= -E₀r\cos θ + \sum_{ℓ=0}^∞ \frac{B_ℓ}{r^{ℓ+1}}P_ℓ(\cos θ)
\end{align*}
By our choice of making use of the superposition principle, we have set the
potential to be zero at the surface, so at $r = R$, the boundary conditions
lets us solve for the values of the $B_ℓ$:
\begin{align*}
    0 &= -E₀R\cos θ + \sum_{ℓ=0}^∞ \frac{B_ℓ}{R^{ℓ+1}}P_ℓ(\cos θ)
\intertext{The Legendre polynomial orthogonality again eliminates all
coefficients except $B₁$.}
    E₀R\cos θ &= \frac{B₁}{R²}\cos θ \\
    B₁ &= E₀R³
\end{align*}
This gives us the solution to the grounded sphere as
\begin{align*}
    V₀(r,θ) &= -E₀r\cos θ \left[ 1 - (\frac{R}{r})³ \right]
\end{align*}

Therefore by superposition of both solutions, the potential in this situation
at all points outside the sphere is
\begin{align}
    \boxed{ V(r,θ) = \frac{Q}{4πε₀r} - E₀r\cos θ \left[ 1 -
        (\frac{R}{r})³ \right] }
\end{align}

%%%%%%%%%%%%%%%%%%%%%%%%%%%%%%%%%%%%%%%%%%%%%%%%%%%%%%%%%%%%%%%%%%%%%%%%%%%%%%%
%%%% Problem 3
%%%%%%%%%%%%%%%%%%%%%%%%%%%%%%%%%%%%%%%%%%%%%%%%%%%%%%%%%%%%%%%%%%%%%%%%%%%%%%%
\problem{3}
\subsubsection{Question}
% Keywords
	\index{quantum!Reflection and transmission through a barrier}

Consider the transmission of a beam of particles of mass $m$ and momentum $p
= ℏk$, in one dimension, incident on a rectangular potential barrier of
height $V₀$ and extending from $x = 0$ to $x = L$, in the special case that
the energy $E$ of the incident particle \emph{is exactly equal} to the
barrier height $V₀$.
\begin{enumerate}[(a)]
    \item
        Calculate the transmission and reflection coefficients $T$ and $R$.
    \item
        Check some properties of your answers in (a): is probability
        conserved? Do $T$ and $R$ have the expected limiting values for $L$
        very large or very small?
    \item
        For what values of the de Broglie wavelength of th eparticles is th
        etransmitted fraction equal to $1/2$?
\end{enumerate}

\subsubsection{Answer}

Consider a beam of particles incident from the left as shown in the figure
below:
\begin{figure}[H]
    \centering
    \begin{tikzpicture}
        % Make the axes
        \draw[dashed,<->] (-1,0) -- (8,0)
            node [anchor=west] {$\hat x$};
        \draw[dashed,<->] (2,-1) -- (2,2.5)
            node [anchor=south] {$V$};
        % Draw the potential
        \draw (0,0) -- (2,0) -- (2,1.5) -- (5,1.5) -- (5,0) -- (7,0);
        % Label the special values of note
        \draw (2,0) node[anchor=north] {$x = 0$};
        \draw (5,0) node[anchor=north] {$x = L$};
        \draw (5,1.5) node[anchor=west] {$V = V₀$};
        % Show the indicent particle beam
        \draw[very thick,->] (0,1.5)
            node[anchor=east] {$E = V₀$}
            -- (1.9,1.5);
        % Label the regions
        \draw (1,   0.75) node {\Large Ⅰ};
        \draw (3.5, 0.75) node {\Large Ⅱ};
        \draw (6,   0.75) node {\Large Ⅲ};
    \end{tikzpicture}
\end{figure}

Ignoring normalization for a minute, we know that in regions Ⅰ and Ⅲ that
the wavefunction is that of a free particle:
\begin{align*}
    ψ(x) &= Ae^{ikx} + B^{-ikx}
        & k² &= \frac{2mE}{ℏ²}
\end{align*}
In region Ⅱ, the energy $E$ cancels with the potential $V$ in the Schrödinger
equation, so the solution takes the form of a first order polynomial
\begin{align*}
    ψ(x) &= Ax + B
\end{align*}

We will only be concerned with the reflection and transmission coefficients,
and knowing that they are defined in terms of a ratio of the wavefunction
amplitude for the reflected and transmitted components with respect to the
incident amplitude, we simplify our solution by directly setting the incident
particle amplitude to unity. Furthermore, we know that there is no leftward
traveling component in region Ⅲ. Assigning each component a unique and
appropriate unknown coefficient, the three wavefunctions are
\begin{align*}
    ψ_{\text{Ⅰ}} &= e^{ikx} + re^{-ikx}\\
    ψ_{\text{Ⅱ}} &= ax + b\\
    ψ_{\text{Ⅲ}} &= te^{ikx}
\end{align*}

We find the values for $r$ and $t$ by applying continuity boundary conditions
at the interfaces between each solution. Starting at $x = 0$,
\begin{align*}
    ψ_{\text{Ⅰ}}(0) &= ψ_{\text{Ⅱ}}(0)
        & ψ_{\text{Ⅰ}}'(0) &= ψ_{\text{Ⅱ}}'(0) \\
    1 + r &= b
        & ik(1-r) &= a
\end{align*}
Then putting the values into $ψ_{\text{Ⅰ}}$ and solving the boundary conditions
at $x = L$,
\begin{align*}
    ψ_{\text{Ⅱ}}(L) &= ψ_{\text{Ⅲ}}(L)
        & ψ_{\text{Ⅱ}}'(L) &= ψ_{\text{Ⅲ}}'(L) \\
    ik(1-r)L + 1+r &= te^{ikL}
        & ik(1-r) &= ikte^{ikL}
\end{align*}
From the equation on the right, we solve for $t$ as a function of $r$ and
insert it into the condition on the left:
\begin{align*}
    t &= (1-r)e^{-ikL} \\
    ik(1-r)L + 1 + r &= ((1-r)e^{-ikL})e^{ikL} \\
    r &= \frac{-ikL}{2-ikL}
\intertext{Plugged back into $t$ gives}
    t &= \frac{2}{2-ikL}e^{-ikL}
\end{align*}

We then just take the complex square of both amplitudes to get the reflection
and transmission coefficients:
\begin{align}
    \boxed{ R = |r|² = \frac{k²L²}{4 + k²L²} }
\end{align}
\begin{align}
    \boxed{ T = |t|² = \frac{4}{4 + k²L²} }
\end{align}

These satisfy the requisite properties: the probabilities sum to unity so all
particles are accounted for, in the limit that the barrier vanishes no
particles are reflected and all are transmitted, and in the limit that the
barrier grows to infinite depth, all particles are reflected.
\begin{align}
    \boxed{ R + T = 1}
\end{align}
\begin{align}
    \boxed{
    T \underset{L→0}{\longrightarrow} 1 \quad\quad
        T \underset{L→∞}{\longrightarrow} 0
    }
\end{align}
\begin{align}
    \boxed{
    R \underset{L→0}{\longrightarrow} 0 \quad\quad
        R \underset{L→∞}{\longrightarrow} 1
    }
\end{align}

This system can be tuned such that half of the particles are transmitted
through the barrier by changing the energy of the particles. To do so, we set
the transmission probability to $\frac 12$ and solve for the particles'
corresponding de Broglie wavelength.
\begin{align*}
    \frac 12 &= \frac{4}{4+k²L²} \\
    4 &= k²L² \\
    k² &= \frac{4}{L²}
\intertext{Making use of the definition of $k²$ in terms of the energy,}
    \frac{2mE}{ℏ²} &= \frac{4}{L²}
\intertext{Then writing the energy in terms of the de Broglie wavelength:}
    \frac{2m}{ℏ²} \frac{4π²ℏ²}{2mλ²} &= \frac{4}{L²}
\end{align*}

Therefore, the particles' incident momentum can be tuned and half the particles
will be transmitted when
\begin{align}
    \boxed{ λ = πL }
\end{align}

%%%%%%%%%%%%%%%%%%%%%%%%%%%%%%%%%%%%%%%%%%%%%%%%%%%%%%%%%%%%%%%%%%%%%%%%%%%%%%%
%%%% Problem 4
%%%%%%%%%%%%%%%%%%%%%%%%%%%%%%%%%%%%%%%%%%%%%%%%%%%%%%%%%%%%%%%%%%%%%%%%%%%%%%%
\problem{4}
\subsubsection{Question}
% Keywords
	\index{quantum!Periodic array of potential wells}
	\index{solid state!Periodic array of potential wells}

Consider a one-dimensional infinite array of points labeled by an index $n$
and separated by a fixed unit distance. At each point there is an identical
very deep and narrow potential well. Let $\ket{n}$ denote an eigenstate of a
\emph{single} well, with energy $E$.
\begin{enumerate}[(a)]
    \item
        Argue that if the wells are so narrow that the different sites can
        be considered uncoupled, then $\ket{n}$ is an eigenstate of the
        total Hamiltonian $H$ with eigenvalue $E$. What is its degeneracy?
        Then show that the state $\ket{k}$ defined as
        \begin{align*}
            \ket{k} = \sum_{n=-∞}^∞ e^{ink} \ket{n}
        \end{align*}
        with $-π < k < π$ is an eigenstate of both $H$ and the translation
        operator $T$ defined as $T\ket{n} = \ket{n+1}$. Find the respective
        eigenvalues.
    \item
        Assume now that neighboring sites are weakly coupled so that the
        total Hamiltonian can now be written as
        \begin{align*}
            H = \sum_{n=-∞}^∞ ( \ket{n}E\bra{n} - \ket{n+1}D\bra{n} -
                \ket{n}D\bra{n+1} )
        \end{align*}
        where the coupling parameter $D$ is real and we assume that
        $\braket{n}{n'} = δ_{n,n'}$. Show that $\ket{n}$ is no longer an
        eigenstate of $H$ but that $\ket{k}$ still is. Find the eigenvalue.
\end{enumerate}

\subsubsection{Answer}

The only reasonable choice for the form of the total Hamiltonian $H$ is a
superposition of the Hamiltonian of individual sites.
\begin{align*}
    H &= \sum_i H_i
\end{align*}
If we then operate on a state $\ket{n}$ with the total Hamiltonian,
\begin{align*}
    H\ket{n} &= (\sum_i H_i)\ket{n} \\
    {} &= \sum_i H_i\ket{n} \\
    H\ket{n} &= H_i δ_{i,n} \ket{n}
\end{align*}
Only the $n$-th Hamiltonian will operate on $\ket{n}$, so the state is in fact
an eigenstate of the total Hamiltonian with an eigenvalue of $E$.
\begin{align}
    \boxed{ H\ket{n} = E\ket{n}} \\
    \boxed{ \text{$N$-fold degeneracy} }
\end{align}
Because each state $n$ has the same eigenvalue of $E$, the degeneracy is
equal to the number of sites. If there are $N$ sites in the array, then that
is also the degeneracy of the total system.

Similarly for the state $\ket{k}$ as defined will can be operated on by the
total Hamiltonian:
\begin{align*}
    H\ket{k} &= H (\sum_n e^{ink}\ket{n}) \\
    {} &= \sum_n e^{ink} H\ket{n}
\intertext{Then because we've already shown that $\ket{n}$ is an eigenstate
of $H$ with eigenvalue $E$}
    {} &= \sum_n e^{ink} E\ket{n} \\
    H\ket{k} &= E (\sum_n e^{ink}\ket{n})
\end{align*}
We find that $\ket{k}$ is also an eigenstate of the total Hamiltonian with
ane eigenvalue of $E$ as well.
\begin{align}
    \boxed{ H\ket{k} = E\ket{k} }
\end{align}

Finally, we define a translation operator $T$ for $\ket{n}$ and determine its
effect on the state $\ket{k}$.
\begin{align*}
    T\ket{k} &= T (\sum_n e^{ink}\ket{n}) \\
    {} &= \sum_n e^{ink} T\ket{n} \\
    {} &= \sum_n e^{ink} \ket{n+1} \\
\intertext{We can insert a factor of unity to extract a more useful form}
    {} &= \sum_n e^{i(n+1)k}e^{-ik} \ket{n+1} \\
    {} &= e^{-ik} \sum_n e^{i(n+1)k} \ket{n+1} 
\end{align*}
and since $n ∈ (-∞,∞)$, the distinction between $n$ and $n+1$ is
inconsequential to the definition of $\ket{k}$. Therefore
\begin{align}
    \boxed{ T\ket{k} = e^{-ik}\ket{k} }
\end{align}
The translation operator has a phase eigenvalue of $e^{-ik}$ when operating
on the Bloch wave function $\ket{k}$.

If the total Hamiltonian is then modified include nearest neighbor
interactions, the individual site wavefunctions $\ket{n}$ are no longer
eigenstates of the total Hamiltonian as shown by explicit calculation:
\begin{align*}
    H\ket{n} &= \left[ \sum_{n'} \ket{n'}E\bra{n'} - \ket{n'+1}D\bra{n'}
        - \ket{n'}D\bra{n'+1} \right] \ket{n} \\
    {} &= \sum_{n'} \ket{n'}E\braket{n'}{n} -
        \ket{n'+1}D\braket{n'}{n} - \ket{n'}D\braket{n'+1}{n} \\
    {} &= \sum_{n'} δ_{n,n'}(E\ket{n'} - D\ket{n'+1}) - Dδ_{n,n'+1}\ket{n'} \\
    {} &= E\ket{n} - D\ket{n+1} - D\ket{n-1}
\end{align*}
There are now three wavefunctions left with two different coeffients, so the
problem is not an eigenvalue problem.
\begin{align}
    \boxed{ E\ket{n} - D\ket{n+1} - D\ket{n-1} = H\ket{n} ≠ λ\ket{n} }
\end{align}

Operating on the Bloch wavefunction, though
\begin{align*}
    H\ket{k} &= \left[ \sum_{n'} \ket{n'}E\bra{n'} - \ket{n'+1}D\bra{n'}
        - \ket{n'}D\bra{n'+1} \right] (\sum_n e^{ink}\ket{n}) \\
    {} &= \sum_{n,n'} \ket{n'}Ee^{ink}\braket{n'}{n} -
        \ket{n'+1}De^{ink}\braket{n'}{n} - \ket{n'}De^{ink}\braket{n'+1}{n} \\
    {} &= \sum_{n,n'} Ee^{ink}δ_{n,n'}\ket{n'} - De^{ink}δ_{n,n'}\ket{n'+1} -
        De^{ink}δ_{n,n'+1}\ket{n'} \\
\intertext{Consuming the summation over $n'$ to select a non-zero term in the
Kronecker delta leaves}
    {} &= \sum_{n} Ee^{ink}\ket{n} - De^{ink}\ket{n+1} - De^{ink}\ket{n-1}
\intertext{We can then separate each term into a summation:}
    {} &= E\sum_{n}(e^{ink}\ket{n}) - D\sum_{n}(e^{ink}\ket{n+1}) -
        D\sum_{n}(e^{ink}\ket{n-1})
\intertext{Then using the same unity-factor trick as in demonstrating that
$\ket{k}$ is an eigenstate of $T$,}
    {} &= E\sum_{n}(e^{ink}\ket{n}) - De^{-ik}\sum_{n}(e^{i(n+1)k}\ket{n+1}) -
        De^{ik}\sum_{n}(e^{i(n-1)k}\ket{n-1})
\end{align*}
Each of these terms is the definition of $\ket{k}$, so we find that $\ket{k}$
is still an eigenstate of this new Hamiltonian with an eigenvalue of $E - 2D
\cos k$.
\begin{align}
    \boxed{ H\ket{k} = (E - 2D\cos k)\ket{k} }
\end{align}

%%%%%%%%%%%%%%%%%%%%%%%%%%%%%%%%%%%%%%%%%%%%%%%%%%%%%%%%%%%%%%%%%%%%%%%%%%%%%%%
%%%% Problem 5
%%%%%%%%%%%%%%%%%%%%%%%%%%%%%%%%%%%%%%%%%%%%%%%%%%%%%%%%%%%%%%%%%%%%%%%%%%%%%%%
\problem{5}
\subsubsection{Question}
% Keywords
	\index{thermodynamics!Mixing gases}

Two monatomic ideal gases, each occupying a volume $V = \SI{1}{\m³}$, are
separated by a removeable insulating partition. They have different
temperatures $T₁ = \SI{350}{\K}$ and $T₂ = \SI{450}{\K}$, and different
pressures $p₁ = \SI{e3}{\N\per\m²}$ and $p₂ = \SI{5e3}{\N\per\m²}$. The
partition is removed, and the gases are allowed to mix while remaining
thermally isolated from the outside.
\begin{enumerate}[(a)]
    \item
        What are the final temperature $T_f$ (in \si{\K}) and pressures $p_f$
        (in \si{\N\per\m²})?
    \item
        What is the net change in entropy due to mixing (in \si{\J\per\K})?
\end{enumerate}

\subsubsection{Answer}

Starting with conservation of energy, the total internal energy after the
partition is removed must be the same as the sum of the internal energies of
both starting gases.
\begin{align*}
    U &= U₁ + U₂ \\
    \frac 32 Nk_B T_f &= \frac 32 N₁k_B T₁ + \frac 32 N₂k_B T₂ \\
    T_f &= \frac{N₁T₁ + N₂T₂}{N₁ + N₂}
\end{align*}
where we made use of the fact that particle number must also be a conserved
quantity so that $N = N₁ + N₂$. The original particle numbers $N₁$ and $N₂$
can be determined from the ideal gas law in the initial state:
\begin{align*}
    p₁V &= N₁k_B T₁             &   p₂V &= N₂ k_B T₂ \\
    N₁ &= \frac{p₁V}{k_B T₁}    &   N₂ &= \frac{p₂V}{k_B T₂}
\end{align*}
Plugging these into the final temperature $T_f$ above and simplifying gives
the value in terms of known quantities as
\begin{align*}
    T_f &= \frac{p₁ + p₂}{p₁T₂ + p₂T₁} T₁T₂
\end{align*}
which when the numbers are plugged in gives a final temperature of
\begin{align}
    \boxed{ T_f = \SI{429.55}{\K} }
\end{align}
We can then plug this temperature into a formulation of the ideal gas law
for the system after the partition has been removed to get the final pressure.
\begin{align*}
    p_f (2V) &= (N₁+N₂) k_B T_f \\
    p_f &= \frac{(N₁+N₂)k_B}{2V} \frac{N₁T₁ + N₂T₂}{N₁+N₂} \\
    p_f &= \frac{k_B}{2V}(N₁T₁ + N₂T₂)
\end{align*}
Again substitution for $N₁$ and $N₂$ in terms of the original pressures and
temperatures gives
\begin{align*}
    p_f &= \frac{p₁ + p₂}{2}
\end{align*}
which when the values are plugged in
\begin{align}
    \boxed{ p_f = \SI{3e3}{\N\per\m²} }
\end{align}

From the Sackur-Tetrode euation, we can calculate the change in the entropy
from the beginning state to the final one. We start by simplifying the
equation to isolate constant factors:
\begin{align*}
    S &= Nk_B \left\{\frac 52 + \ln\left[ \frac{V}{N}
		(\frac{4πmU}{3Nh²})^{3/2} \right] \right\} \\
    {} &= \frac 52 Nk_B + Nk_B \ln\left[ \frac{V}{N}
		(\frac{4πmU}{3Nh²})^{3/2} \right] \\
    {} &= Nk_B \left\{ \frac 52 + \frac 32 \ln(\frac{4πm}{3h²})
        + \ln\left[ \frac{V}{N}(\frac{U}{N})^{3/2} \right] \right\}
\end{align*}
Then with the change in entropy defined as
\begin{align*}
    ΔS = S - S₁ - S₂
\end{align*}
we start calculating the sum term-by-term. For the first two constant terms,
the fact that $N = N₁ + N₂$ causes these terms to cancel with those in $S₁$
and $S₂$. That leaves us just with the last term.
\begin{align*}
    ΔS &= (N₁+N₂)k_B
        \ln\left[\frac{2V}{N}(\frac{\frac 32 Nk_B T_f}{N})^{3/2}\right]
        - N₁k_B\ln\left[\frac{V}{N₁}(\frac{\frac 32 N₁k_B T₁}{N₁})^{3/2}\right]
        - N₂k_B\ln\left[\frac{V}{N₂}(\frac{\frac 32 N₂k_B T₂}{N₂})^{3/2}\right]
    \\
\intertext{By the ideal gas law, $V/N = k_BT/p$ which simplifies the expression
to}
    {} &= N₁k_B\ln\left[
            \frac{2k_B T_f}{p_f}(\frac 32 k_B T_f)^{3/2} ⋅
            \frac{p₁}{k_B T₁}(\frac{1}{\frac 32 k_B T₁})^{3/2}
        \right] + N₂k_B\ln\left[
            \frac{2k_B T_f}{p_f}(\frac 32 k_B T_f)^{3/2} ⋅
            \frac{p₂}{k_B T₂}(\frac{1}{\frac 32 k_B T₂})^{3/2}
        \right]
    \\
    {} &= N₁k_B\ln\left[
            \frac{p₁}{p_f}(\frac{T_f}{T₁})^{5/2}
        \right] + N₂k_B\ln\left[
            \frac{p₂}{p_f}(\frac{T_f}{T₂})^{5/2}
        \right] + (N₁+N₂)k_B \ln 2
\end{align*}
Using the ideal gas law again to manipulate the coefficients $Nk_B = pV/T$,
\begin{align*}
    ΔS &= V\left\{ \frac{p₁}{T₁}\ln\left[
            \frac{p₁}{p_f}(\frac{T_f}{T₁})^{5/2}
        \right] + \frac{p₂}{T₂}\ln\left[
            \frac{p₂}{p_f}(\frac{T_f}{T₂})^{5/2}
        \right] + \frac{p_f}{T_f} \ln 2 \right\}
\end{align*}
Plugging in the values for all these numbers as given or we determined, the
change in entropy is
\begin{align}
    \boxed{ ΔS = \SI{7.55}{\J/\K} }
\end{align}

%%%%%%%%%%%%%%%%%%%%%%%%%%%%%%%%%%%%%%%%%%%%%%%%%%%%%%%%%%%%%%%%%%%%%%%%%%%%%%%
%%%% Problem 6
%%%%%%%%%%%%%%%%%%%%%%%%%%%%%%%%%%%%%%%%%%%%%%%%%%%%%%%%%%%%%%%%%%%%%%%%%%%%%%%
\problem{6}
\subsubsection{Question}
% Keywords
	\index{special relativity!Rocket sending signal to Earth}

A rocket passes Earth at a speed $v = 0.6c$. When a clock on the rocket says
that one hour has elapsed since passing, the rocket sends a light signal
back to Earth.
\begin{enumerate}[(a)]
    \item
        Suppose that the Earth and rocket clocks were synchronized at zero
        at the time passing. According to the \emph{Earth} clocks, when was
        the signal sent?
    \item
        According to the \emph{Earth} clocks, when did the signa arrive back
        on Earth?
    \item
        According to the \emph{rocket} clocks, how long after the rocket
        passed did the signal arrive back on Earth?
\end{enumerate}

\subsubsection{Answer}

Let $β = v/c$.

\begin{enumerate}[(a)]
    \item
        Let $t'_{sent} = \SI{1}{\hour}$ be the time at which the rocket sent
        the signal according to its own clock. Because from the Earth's
        reference frame the rocket's clocks are running slow, the clocks on
        Earth must show a time greater than the rocket's by a factor of $γ$.
        \begin{align*}
            t_{sent} &= γt'_{sent} \\
            {} &= \frac{\SI{1}{\hour}}{\sqrt{1 - β²}}
        \end{align*}
        \begin{align}
            \boxed{ t_{sent} = \frac 54 \,\si{\hour} = \SI{1.25}{\hour} }
        \end{align}
    \item
        The returning light will traverse the intermediate distance at $c$, so
        $t_{ret} = x/c$. The distance from the Earth is simply the velocity
        times the time (in Earth's reference frame), so together,
        \begin{align*}
            t_{ret} &= \frac{vt_{sent}}{c} \\
            {} &= βt_{sent} \\
            t_{ret} &= \frac 34 \,\si{\hour} = \SI{0.75}{\hour}
        \end{align*}
        The total round trip time is then
        \begin{align}
            \boxed{ t_{tot} = t_{sent} + t_{ret} = \SI{2}{\hour} }
        \end{align}
    \item
        Because the rocket is flying away from the Earth, the distance behind
        it towards the earth appears to have been length expanded by a factor
        of $γ$ compared to the distance that Earth would report. Therefore the
        time for the signal to be sent from the rocket to Earth appears to the
        rocket to be
        \begin{align*}
            t'_{ret} &= \frac{γx}{c} \\
            {} &= γt_{ret} \\
            t'_{ret} &= \frac{15}{16} \,\si{\hour} = \SI{0.9375}{\hour}
        \end{align*}
        Added to the 1 hour that the rocket observes as the time before it
        sent the signal, the signal would arrive at earth at time
        \begin{align*}
            \boxed{ t'_{tot} = \frac{31}{16} \,\si{\hour} = \SI{1.9375}{\hour} }
        \end{align*}
\end{enumerate}
