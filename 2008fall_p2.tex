%%%%%%%%%%%%%%%%%%%%%%%%%%%%%%%%%%%%%%%%%%%%%%%%%%%%%%%%%%%%%%%%%%%%%%%%%%%%%%%
%%%% Problem 1
%%%%%%%%%%%%%%%%%%%%%%%%%%%%%%%%%%%%%%%%%%%%%%%%%%%%%%%%%%%%%%%%%%%%%%%%%%%%%%%
\problem{1}
\subsubsection{Question}
% Keywords
	\index{mechanics!Bead on a hoop}

A particle of mass $m$ is constrained to move without friction on a circular
wire of radius $R$ rotating with constant angular frequency $ω$ about a
vertical diameter. Gravity can not be neglected.
\begin{enumerate}[a)]
    \item
        Write down the Lagrangian for the system and the equations of motion.
    \item
        Find the equilibrium position(s) of the particle and determine
        whether this position is stable.
    \item
        Calculate the frequency of small oscillations about any stable points.
\end{enumerate}

\begin{figure}[H]
    \centering
    \begin{tikzpicture}
        % The axis
        \draw [dashed,->] (0,-1) -- (0,4);
        % The hoop
        \draw (0,1.5) circle (1.5);
        % The bead
        \draw ($(0,1.5)!1.5cm!30:(0,0)$)
            node[circle,fill=black,anchor=center,inner sep=2pt] {}
            coordinate (bead);
        % Label the angle
        \draw [dashed] (0,1.5) -- (bead);
        \draw ($(0,1.5)!0.5cm!(0,0)$)
            coordinate (arc start)
            arc (-90:-60:0.5cm)
            coordinate (arc end);
        \draw ($(arc start)!0.5!(arc end)$)
            node [anchor=north] {$\,\,θ$};
    \end{tikzpicture}
\end{figure}

\subsubsection{Answer}

To start constructing the Lagrangian and equations of motion, we first specify
the kinetic and potential energies. For the kinetic energy, there is an energy
associated with the rotation about the axis and one along the bead. These
combined to give
\begin{align*}
    T &= \frac 12 m (Rω\sin θ)² + \frac 12 m (R\dot θ)² \\
    {} &= \frac 12 m R² ω² \sin² θ + \frac 12 m R² {\dot θ}²
\end{align*}
The potential energy is all gravitational, so
\begin{align*}
    V &= -mgR\cos θ
\end{align*}
where the zero point was taken to be at the center of the hoop to avoid adding
extra constant terms to the Lagrangian. Combining the two, we get
\begin{align}
    \boxed{
    \mathcal L = \frac 12 mR²ω²\sin² θ + \frac 12 mR²{\dot θ}² + mgR\cos θ
    }
\end{align}
Taking the appropriate derivatives in $θ$, the equation of motion is
\begin{align}
    \boxed{
    \ddot θ = ω² \sin θ \cos θ - \frac{g}{R}\sin θ
    }
\end{align}

In order to determine any possible stable points, we note that a stable point
is a place where the angle does not change in time. Since this also equates to
$\ddot θ = 0$, we set the equation above equal to zero and solve for the angles
which satisfy this condition. They end up being the trivial $θ = \{0, π\}$
where the sine function is zero as well as
\begin{align*}
    \cos θ₀ &= \frac{g}{Rω²}
\end{align*}
The three stable points are then
\begin{align}
    \boxed{ θ₀ = \left\{ 0, \arccos(\frac{g}{Rω²}), π \right\} }
\end{align}

To determine the stability of each, we must determine whether we get
oscillatory or exponential solutions to the differential equation of motion.
To do this, we suppose the angle $θ$ is composed of the equilibrium angle
$θ₀$ and a small perturbation $δ$. Expanding the equation in terms of this,
\begin{align*}
    \ddot δ = ω²\sin(θ₀+δ)\cos(θ₀+δ) - \frac{g}{R}\sin(θ₀+δ)
\end{align*}
Using several trigonometric expansions, the equation can be expanded into the
form
\begin{align*}
    \ddot δ = ω²\left[ \cos θ₀\sin θ₀ (\cos²δ - \sin²δ) + \cos δ\sin δ
        (\cos²θ₀ - \sin²θ₀) \right] - \frac{g}{R}\left[ \sin θ₀\cos δ +
        \cos θ₀\sin δ \right]   
\end{align*}

For the case where $θ₀ = 0$,
\begin{align*}
    \ddot δ &= ω² \cos δ \sin δ - \frac{g}{R} \sin δ
\intertext{Expanding to first order in $δ ≈ 0$,}
    \ddot δ &= -(\frac{g}{R} - ω²)δ
\end{align*}
Therefore, the equilibrium point $θ₀ = 0$ is only stable if $ω <
\sqrt{\frac{g}{R}}$.

Likewise for for $θ₀ = π$,
\begin{align*}
    \ddot δ &= (\frac{g}{R} + ω²) δ
\end{align*}
The coefficient on $δ$ will never be negative, so the angle $θ₀ = π$ will
be unstable under all conditions.

For the final angle where $θ₀ = \arccos(\frac{g}{Rω²})$, we must do several
substitutions and expansions. $\cos θ₀$ is trivial. $\sin θ₀$ ends up being
$\sqrt{Rω² - g²}/(Rω²)$ by triangle relations. If we substitute these in plus
do an expansion to first order for small $δ$, we get the equation
\begin{align*}
    \ddot δ &= ω² \left[ \frac{g\sqrt{Rω²-g²}}{R²ω²} + δ \frac{2g²-Rω²}{R²ω²}
        \right] - \frac{g}{R} \left[\frac{\sqrt{Rω²-g²}}{R²ω²} + δ
        \frac{g}{Rω²} \right]
\end{align*}
If we consider only the homogeneous terms dependent on $δ$,
\begin{align*}
    \ddot δ &= \frac{g² - Rω²}{R²ω²} δ
\end{align*}
This equation is stable if and only if the coefficient on $δ$ is negative, so
it must be that $ω > \frac{g}{\sqrt{R}}$.

In summary, the equilibrium points have the following conditions:
\begin{align}
    \boxed{ θ₀ = 0 \quad\quad \text{Stable iff } ω < \sqrt{\frac gR} }
\end{align}
\begin{align}
    \boxed{ θ₀ = \arccos(\frac{g}{Rω²}) \quad\quad \text{Stable iff }
        ω > \frac{g}{\sqrt{R}} }
\end{align}
\begin{align}
    \boxed{ θ₀ = π  \quad\quad \text{Never stable} }
\end{align}


About the two stable points, we simply use the coefficient that has already
been isolated to determine the frequency of the oscillations about that point.
\begin{align}
    \boxed{ ω₁ = \sqrt{\frac{g}{R} - ω²} \quad\quad\text{for $θ₀ = 0$} }
\end{align}
\begin{align}
    \boxed{ ω₂ = \sqrt{\frac{Rω² - g²}{R²ω²}}\quad\quad\text{for $θ₀ =
        \arccos(\frac{g}{Rω²})$} }
\end{align}

%%%%%%%%%%%%%%%%%%%%%%%%%%%%%%%%%%%%%%%%%%%%%%%%%%%%%%%%%%%%%%%%%%%%%%%%%%%%%%%
%%%% Problem 2
%%%%%%%%%%%%%%%%%%%%%%%%%%%%%%%%%%%%%%%%%%%%%%%%%%%%%%%%%%%%%%%%%%%%%%%%%%%%%%%
\problem{2}
\subsubsection{Question}
% Keywords
	\index{electrostatics!Charged sphere in uniform electric field}
    \index{Gauss' Law!Charged sphere in uniform electric field}

The general solution of the Laplace's equation for an electrostatic problem
having azimuthal symmetry can be written as
\begin{align*}
    V(r,θ) &= \sum_{ℓ=0}^∞ (A_ℓ r^ℓ + \frac{B_ℓ}{r^{ℓ+1}}) P_ℓ(\cos θ)
\end{align*}
Now consider the follwoing problem. A solid spherical conductor of radius $R$
having charge $Q$ is placed in an otherwise uniform electric field $\vec E
= E₀\hat z$.
\begin{enumerate}[(a)]
    \item
        Qualtitatively describe the electric field inside and outside of the
        sphere.
    \item
        Solve the problem and find the electric potential in th eregion outside
        the sphere.
\end{enumerate}

\subsubsection{Answer}

To provide a qualitative description, we make use of several properties of
conductors. The electric field inside the conductor is guaranteed to be zero
in the limit of a perfect conductor which can move its electrons anywhere
they're needed to cancel any applied fields. For the region outside the
sphere, it is easiest to describe the region just outside the surface and
the region at infinitely large distances. Far away, the effects of the the
sphere are negligible and the electric field is the external uniform field.
Near the surface, though, all field lines are perpendicular to the surface;
therefore, the external field's lines are curved so that any intersections
occur perpendicular to the surface.

In summary
\par\fbox{\begin{minipage}{\textwidth}
    \begin{enumerate}
        \item
            The field is uniform at large distances
        \item
            The field is perpendicular to the surface of the conductor at
            the conductor's surface
        \item
            There is no field within the interior of the conductor
    \end{enumerate}
\end{minipage}}
\vspace{\baselineskip}

In order to analyze the problem analytically, we make use of the
superposition principle to simplify the problem. Because the sphere carries
its own charge, we treat this case as a superposition of the two simpler
cases of a charged sphere in vacuum and that of a perfectly conducting,
grounded sphere in a uniform electric field.

Since we are only concerned with the potential outside the sphere, we can use
Gauss' Law to get the potential due to the charge $Q$. It is
\begin{align*}
    V_Q(r,θ) &= \frac{Q}{4πε₀r} & r &> R
\end{align*}

The uniform field is the considered by satisfying the appropriate boundary
conditions to solve for the coefficients $A_ℓ$ and $B_ℓ$ in the general
solution given above. We start by converting the given electric field to a
potential. In Cartesian coordinates,
\begin{align*}
    \vec E = E₀\hat z = -\vec ∇V_∞ \quad\quad\Rightarrow\quad\quad V_∞ &= -E₀z
\end{align*}
which when converted to spherical coordinates gives the potential as $r→∞$
\begin{align*}
    V_∞ &= -E₀r\cos θ
\end{align*}
In the infinite distance limit, $B_ℓ/r^{ℓ+1} → 0$ so the boundary condition
equation becomes
\begin{align*}
    -E₀r\cos θ &= \sum_{ℓ=0}^∞ A_ℓ r^ℓ P_ℓ(\cos θ) \\
\intertext{By the orthogonality of the Legendre polynomials, only $A₁$ is
non-zero:}
    -E₀r\cos θ &= A₁r\cos θ \\
    A₁ &= -E₀
\end{align*}

The general solution has thus been simplified to
\begin{align*}
    V₀(r,θ) &= -E₀r\cos θ + \sum_{ℓ=0}^∞ \frac{B_ℓ}{r^{ℓ+1}}P_ℓ(\cos θ)
\end{align*}
By our choice of making use of the superposition principle, we have set the
potential to be zero at the surface, so at $r = R$, the boundary conditions
lets us solve for the values of the $B_ℓ$:
\begin{align*}
    0 &= -E₀R\cos θ + \sum_{ℓ=0}^∞ \frac{B_ℓ}{R^{ℓ+1}}P_ℓ(\cos θ)
\intertext{The Legendre polynomial orthogonality again eliminates all
coefficients except $B₁$.}
    E₀R\cos θ &= \frac{B₁}{R²}\cos θ \\
    B₁ &= E₀R³
\end{align*}
This gives us the solution to the grounded sphere as
\begin{align*}
    V₀(r,θ) &= -E₀r\cos θ \left[ 1 - (\frac{R}{r})³ \right]
\end{align*}

Therefore by superposition of both solutions, the potential in this situation
at all points outside the sphere is
\begin{align}
    \boxed{ V(r,θ) = \frac{Q}{4πε₀r} - E₀r\cos θ \left[ 1 -
        (\frac{R}{r})³ \right] }
\end{align}
