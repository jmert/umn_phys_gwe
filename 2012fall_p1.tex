%%%%%%%%%%%%%%%%%%%%%%%%%%%%%%%%%%%%%%%%%%%%%%%%%%%%%%%%%%%%%%%%%%%%%%%%%%%%%%%
%%%% Problem 1
%%%%%%%%%%%%%%%%%%%%%%%%%%%%%%%%%%%%%%%%%%%%%%%%%%%%%%%%%%%%%%%%%%%%%%%%%%%%%%%
\subsection{Problem 1}
\subsubsection{Question}
Assume you have three identical particles and three single particle states
$|a〉$, $|b〉$, and $|c〉$ available for them. Count how many different
three-particle states there can be if the particles are (a) fermions and (b)
bosons.

%%%%%%%%%%%%%%%%%%%%%%%%%%%%%%%%%%%%%%%%%%%%%%%%%%%%%%%%%%%%%%%%%%%%%%%%%%%%%%%
%%%% Problem 2
%%%%%%%%%%%%%%%%%%%%%%%%%%%%%%%%%%%%%%%%%%%%%%%%%%%%%%%%%%%%%%%%%%%%%%%%%%%%%%%
\clearpage
\subsection{Problem 2}
\subsubsection{Question}
Show that a particle in a one-dimensional infinite square well initially in a
state $Ψ(x,0)$ will always return to that state after a time $T = 4ma²/π\hbar$
where $a$ is the width of the well.

\subsubsection{Answer}
Use the standard time independent Schödinger equation
\begin{align*}
	Ψ(x,t) &= ψ(x) e^{iEt/\hbar}
\end{align*}
with associated differential equation
\begin{align*}
	-\frac{\hbar²}{2m} \frac{d²ψ}{dx²} + V(x)ψ &= Eψ
\end{align*}

For an infinite square well, the potential has the form
\begin{align*}
	V(x) &=
		\begin{cases}
			0	&	|x| < \frac{a}{2} \\
			∞	&	\text{otherwise}
		\end{cases}
\end{align*}
so that the only region to consider is $-\frac{a}{2} < x < \frac{a}{2}$. In this
region, the differential equation takes the form of a harmonic oscillator
\begin{align*}
	\frac{d²ψ}{dx²} &= -\frac{2mE}{\hbar²}ψ
\end{align*}
leading to solutions
\begin{align*}
	ψ(x) ={}& A\cos kx + B\sin kx \\
	{}& \text{where } k² = \frac{2mE}{\hbar²}
\end{align*}

The boundary conditions $ψ(-\frac{a}{2}) = 0$ and $ψ(\frac{a}{2}) = 0$ impose
\begin{align*}
	ψ\left(-\frac{a}{2}\right) &= 0 = A\cos \frac{ka}{2} - B\sin \frac{ka}{2} \\
	ψ\left(\frac{a}{2}\right)  &= 0 = A\cos \frac{ka}{2} + B\sin \frac{ka}{2}
\end{align*}
so that $B = 0$ and
\begin{align*}
	0 &= 2A \cos \frac{ka}{2} \\
	\frac{(2n+1)π}{2} &= \frac{ka}{2} \\
	k &= \frac{(2n+1)π}{a}
\end{align*}

We already had a relation for $k$ defined, so substitute and solve for the
energies $E_n$.
\begin{align*}
	\frac{(2n+1)²π²}{a²} &= \frac{2mE}{\hbar²} \\
	E_n &= \frac{(2n+1)²π²\hbar²}{2ma²}
\end{align*}

Then considering $Ψ(x,t)$, the complex exponential is periodic in time with
period
\begin{align*}
	T_n &= \frac{2π\hbar}{E}
\end{align*}
where $n = 0$ will be the case with the longest periodicity, so
\begin{align*}
	T &= \frac{2π\hbar · 2ma²}{π²\hbar²} \\
	{} &= \frac{4ma²}{π\hbar}
\end{align*}

Therefore, the function is periodic in time with a periodicity
\begin{empheq}[box=\fbox]{align}
	T &= \frac{4ma²}{π\hbar}
\end{empheq}

%%%%%%%%%%%%%%%%%%%%%%%%%%%%%%%%%%%%%%%%%%%%%%%%%%%%%%%%%%%%%%%%%%%%%%%%%%%%%%%
%%%% Problem 3
%%%%%%%%%%%%%%%%%%%%%%%%%%%%%%%%%%%%%%%%%%%%%%%%%%%%%%%%%%%%%%%%%%%%%%%%%%%%%%%
\clearpage
\subsection{Problem 3}
\subsubsection{Question}
Consider a system of $N$ independent classical molecules, each at a fixed
position, with magnetic moment $\vec μ$ in an external magnetic field $\vec B$.
Determine the partition function, and hence find the free energy and the
magnetization at temperature $T$, when the molecules can only be oriented
parallel or antiparallel to the external magnetic field.

%%%%%%%%%%%%%%%%%%%%%%%%%%%%%%%%%%%%%%%%%%%%%%%%%%%%%%%%%%%%%%%%%%%%%%%%%%%%%%%
%%%% Problem 4
%%%%%%%%%%%%%%%%%%%%%%%%%%%%%%%%%%%%%%%%%%%%%%%%%%%%%%%%%%%%%%%%%%%%%%%%%%%%%%%
\clearpage
\subsection{Problem 4}
\subsubsection{Question}
A photon collides with a stationary electron. If the photon scatters at an
angle $θ$, show that the resulting wavelength $λ'$ is given in terms of the
original wavelength $λ$ by
\begin{align*}
	λ' &= λ + \frac{h}{mc} (1 - \cos θ)
\end{align*}
where $m$ is the mass of the electron.

\subsubsection{Answer}
Start by considering conservation of momentum for the system. The initial values
are
\begin{align*}
	p_{γx} &= \frac{h}{λ}		&		p'_{γx} &= \frac{h}{λ'} \cos θ \\
	p_{γy} &= 0					&		p'_{γy} &= \frac{h}{λ'} \sin θ \\
	p_{ex} &= 0					&		p'_{ex} &= ? \\
	p_{ey} &= 0					&		p'_{ey} &= ?
\end{align*}
and considering each component in turn:
\begin{align*}
	\frac{h}{λ} + 0 &= \frac{h}{λ'}\cos θ + p'_{ex}
		& 0 &= \frac{h}{λ'}\sin θ + p'_{ey}
	\\
	p'_{ex} &= \frac{h}{λ} - \frac{h}{λ'}\cos θ
		& p'_{ey} &= -\frac{h}{λ'}\sin θ
\end{align*}
The total momentum of the electron is then
\begin{align}
	p²_e &= \left( \frac{h}{λ} - \frac{h}{λ'}\cos θ \right)² +
		\left( -\frac{h}{λ'}\sin θ \right)²
		\nonumber
	\\
	{} &= \frac{h²}{λ²} - \frac{2h²}{λλ'}\cos θ + \frac{h²}{λ'^2}\cos ²θ +
		\frac{h²}{λ'^2}\sin ²θ
		\nonumber
	\\
	p²_e &= h² \left( \frac{1}{λ²} + \frac{1}{λ'^2} \right) -
		\frac{2h²}{λλ'}\cos θ
\end{align}

Then consider energy conservation, with initial values
\begin{align*}
	E_γ &= \frac{hc}{λ}			&		E'_γ &= \frac{hc}{λ'} \\
	E_e &= mc²					&		E'_e &= \frac{{p'_e}²}{2m} + mc²
\end{align*}
leading to the equation
\begin{align*}
	\frac{hc}{λ} + mc² &= \frac{hc}{λ'} + \frac{{p'_e}²}{2m} + mc²
	\\
	\frac{hc}{λ} &= \frac{hc}{λ'} + \frac{h²}{2m} \left( \frac{1}{λ²} +
		\frac{1}{λ'^2} \right) - \frac{2h²}{2mλλ'}\cos θ
	\\
	\frac{hc}{λ} - \frac{hc}{λ'} &= \frac{h²}{2m} \left( \frac{1}{λ²} +
		\frac{1}{λ'^2} \right) - \frac{2h²}{2mλλ'}\cos θ
	\\
	\frac{λ' - λ}{λλ'} &= \frac{h}{2mc} \frac{λ'^2 + λ²}{λ²{λ'}^2} -
		\frac{h}{mcλλ'}\cos θ
	\\
	λ' - λ &= \frac{h}{2mc} \left( \frac{(λ' - λ)² + 2λλ'}{λλ'} \right) -
		\frac{h}{mc}\cos θ
	\\
	λ' - λ &= \frac{h}{2mc} \left( \frac{(λ' - λ)²}{λλ'} + 2 \right) -
		\frac{h}{mc}\cos θ
	\\
	λ' - λ &= \frac{h}{2mc} \frac{(λ' - λ)²}{λλ'} + \frac{h}{mc}(1 - \cos θ)
\end{align*}
The difference in the wavelengths is small, so
\begin{align*}
	\frac{(λ' - λ)²}{λλ'} &≈ 0
\end{align*}
leading to the final Compton scattering equation
\begin{empheq}[box=\fbox]{align}
	λ' &= λ + \frac{h}{mc}(1 - \cos θ)
\end{empheq}

%%%%%%%%%%%%%%%%%%%%%%%%%%%%%%%%%%%%%%%%%%%%%%%%%%%%%%%%%%%%%%%%%%%%%%%%%%%%%%%
%%%% Problem 5
%%%%%%%%%%%%%%%%%%%%%%%%%%%%%%%%%%%%%%%%%%%%%%%%%%%%%%%%%%%%%%%%%%%%%%%%%%%%%%%
\clearpage
\subsection{Problem 5}
\subsubsection{Question}
The composition of a glass block varies as a function of the distance $x$ from
the top surface. Thus the index of refraction $n(x)$ increases as a function of
$x$ according to the relationship $n(x) = 1.50 - (0.20)/(x+1)²$ where $x$ is
expressed in centimeters. A beam of light strikes the surface with an angle of
incidence $θ_i$, measured from the vertical. What will be the direction of the
beam deep inside the block?

%%%%%%%%%%%%%%%%%%%%%%%%%%%%%%%%%%%%%%%%%%%%%%%%%%%%%%%%%%%%%%%%%%%%%%%%%%%%%%%
%%%% Problem 6
%%%%%%%%%%%%%%%%%%%%%%%%%%%%%%%%%%%%%%%%%%%%%%%%%%%%%%%%%%%%%%%%%%%%%%%%%%%%%%%
\clearpage
\subsection{Problem 6}
\subsubsection{Question}
You want to measure a current when using a voltmeter and a precision resistor.
You measure a voltage of \SI{0.85}{\V} across a \SI{10000}{\ohm} resistor and
\SI{-0.05}{\V} when the input to the voltmeter is short-circuited. The precision
of the voltmeter is \SI{0.001}{\V} and the resistor is rated at
\SI{0.1}{\percent}. What is the value of the current and the precision of the
measurement?

%%%%%%%%%%%%%%%%%%%%%%%%%%%%%%%%%%%%%%%%%%%%%%%%%%%%%%%%%%%%%%%%%%%%%%%%%%%%%%%
%%%% Problem 7
%%%%%%%%%%%%%%%%%%%%%%%%%%%%%%%%%%%%%%%%%%%%%%%%%%%%%%%%%%%%%%%%%%%%%%%%%%%%%%%
\clearpage
\subsection{Problem 7}
\subsubsection{Question}
Find the capacitance per unit length of two coaxial conducting cylindrical tubes
or radii $a$ and $b$.

%%%%%%%%%%%%%%%%%%%%%%%%%%%%%%%%%%%%%%%%%%%%%%%%%%%%%%%%%%%%%%%%%%%%%%%%%%%%%%%
%%%% Problem 8
%%%%%%%%%%%%%%%%%%%%%%%%%%%%%%%%%%%%%%%%%%%%%%%%%%%%%%%%%%%%%%%%%%%%%%%%%%%%%%%
\clearpage
\subsection{Problem 8}
\subsubsection{Question}
Suppose that the radius of the Earth were to gravitationally collapse uniformly
by one percent, with its mass remaining the same. What would happen to the
Earth's kinetic energy of rotatio? If it changes, how does it change and by how
much? Assume that the Earth is a uniform sphere.

%%%%%%%%%%%%%%%%%%%%%%%%%%%%%%%%%%%%%%%%%%%%%%%%%%%%%%%%%%%%%%%%%%%%%%%%%%%%%%%
%%%% Problem 9
%%%%%%%%%%%%%%%%%%%%%%%%%%%%%%%%%%%%%%%%%%%%%%%%%%%%%%%%%%%%%%%%%%%%%%%%%%%%%%%
\clearpage
\subsection{Problem 9}
\subsubsection{Question}
A neutron star has a radius of \SI{10}{\km}, a mass of \SI{3e30}{\kg}. Find the
nearest distance to the surface that a person \SI{2}{\m} tall could approach the
pulsar without being pulled apart. Assume a uniform mass distribution, feet
toward the pulsar and that a person starts to come apart when the force that
each half of the body exerts on the other exceeds ten times the body weight on
Earth. What is the period of revolution in a circular orbit about the pulsar at
this distance?

%%%%%%%%%%%%%%%%%%%%%%%%%%%%%%%%%%%%%%%%%%%%%%%%%%%%%%%%%%%%%%%%%%%%%%%%%%%%%%%
%%%% Problem 10
%%%%%%%%%%%%%%%%%%%%%%%%%%%%%%%%%%%%%%%%%%%%%%%%%%%%%%%%%%%%%%%%%%%%%%%%%%%%%%%
\clearpage
\subsection{Problem 10}
\subsubsection{Question}
Given that the latent heat of fusion for water is \SI{334}{\kilo\J\per\kg} and
the density of water and ice are \SI{1000}{\kg\per\m\cubed} and
\SI{917}{\kg\per\m\cubed}, respectively. At what pressure will ice melt at
\SI{-0.1}{\celsius}? It may be sueful to remember that the Gibbs free energy is
the same across a phase boundary.

