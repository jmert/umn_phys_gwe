\documentclass[avery5371]{flashcards}
\usepackage{etoolbox}

\usepackage{amsmath,amssymb}
%% Make some amsmath fixes
%
% As of 2012-12-23 (XeTeX version 3.1415926-2.4-0.9998, TeXLive 2012), XeTeX
% does not properly handle parentheses when using the TeX primitives
%   \overwithdelims
%   \atopwithdelims
%   \abovewithdelims
% Because of this, the macro \binom which internally uses these these commands
% creates a binomial with the wrong size of brackets (too small). Patch around
% this issue by using \left( and \right) around a bracket-less binomial.
% LuaTeX doesn't seem to suffer from this.
	\renewrobustcmd{\binom} [2]{\left(\genfrac{}{}{0pt}{}{#1}{#2}\right)}
	\renewcommand  {\dbinom}[2]{\left(\genfrac{}{}{0pt}{0}{#1}{#2}\right)}
	\renewcommand  {\tbinom}[2]{\left(\genfrac{}{}{0pt}{1}{#1}{#2}\right)}
	
\usepackage[
	exponent-product     = \cdot,
	load-configurations  = abbreviations,
	inter-unit-separator = {}\cdot{},
	per-mode             = symbol-or-fraction,
	separate-uncertainty = true,
	math-micro           = \text{µ},
	text-micro           = µ
]{siunitx}

\XeTeXinputnormalization=1
\usepackage{fontspec}
\usepackage{unicode-math}
	% Set font features which should be true for all fonts loaded by default
	\defaultfontfeatures{
		% Permit using TeX-style ligatures (i.e. --- for em-dash)
		Ligatures = TeX
	}

	% Also allow math to input using Unicode by default
	\usepackage{unicode-math}
	% Use the XITS Math font by default
	\setmathfont[
			% Use italic fonts at all times
			math-style = ISO,
			bold-style = ISO,
			% Force \epsilon,\varepsilon and \phi,\varphi match the unicode
			% character ε,ϵ and φ,ϕ
			vargreek-shape = unicode
		]{XITS Math}
	% Additionally, use an alternate stylistic set for caligraphic letters so
	% that they are different from the script letters
	\setmathfont[range={\mathcal,\mathbfcal},StylisticSet=1]{XITS Math}
	% Finally, request that certain symbols be provided by Latin Modern
	% Math instead of XITS Math since I think the LM ones looks much nicer.
	\setmathfont[range={"220F}]{Latin Modern Math}	% Product
	\setmathfont[range={"2210}]{Latin Modern Math}	% Coproduct
	\setmathfont[range={"2211}]{Latin Modern Math}	% Summation

	% Have the Linux Libertine family of fonts provide the default serif and
	% sans-serif fonts
	\setmainfont{Linux Libertine O}
	\setsansfont{Linux Biolinum O}

%% Make all parentheses stretchy
%
% By default, all parentheses are just normal parentheses in math mode. This
% to me doesn't seem to make sense since I almost always want them to be
% stretchy.
%
% Adapted from http://tex.stackexchange.com/questions/7359/how-to-make-a-real-backslash-escape-character/7372#7372
	% Start by defining a macro to quickly reset the default behavior if we
	% decide that a given environment shouldn't use the active definition. We
	% need the \edef so that the \the\mathcode is immediately expanded before
	% we change things.
	\edef\resetparens{%
			\mathcode`(=\the\mathcode`(
			\mathcode`)=\the\mathcode`)
		}
	% Now we actually do the magic like this:
	%   1) Start a local group so that we can temporarily define a lowercase
	%      tilde to be the left paranetheses and not affect anything later in
	%      the document.
	%      a) We use the tilde because it is already defined to be an active
	%         character. This saves a step since to define a character to
	%         execute a macro sequence, it first needs to be active. The
	%         \lowercase command does for us since the lowercase tilde (i.e.
	%         paren) is made to have the same category code as its uppercase
	%         counterpart.
	%   2) \lccode assigns the lowercase tilde to be the left paren
	%   3) Then use \lowercase to cause us to define the macro of active left
	%      paren (without explicitly changing category codes).
	%      a) The \endgroup ends the local group we just started, but not
	%         before lowercase has already learned how to convert the tilde
	%         This lets the next \edef to be at the current global scope,
	%         avoiding use of \global
	%   4) Then define the active left paren to be the sequence \left( so that
	%      we invoke the stretchy code
	%   5) Finally change the mathcode of the left paren to be active within
	%      math environments, meaning we don't have to worry about the
	%      character being active in normal text mode.
	\begingroup%
			\lccode`~=`(%
			\lowercase{\endgroup\edef~}{\left(}
		\mathcode`(="8000
	% Do it again for the right paren
	\begingroup%
			\lccode`~=`)%
			\lowercase{\endgroup\edef~}{\right)}
		\mathcode`)="8000
	% Finally, this change of math codes breaks some changes the amsmath
	% package makes to the definitions of \left( and \right), so we fix this
	% by patching the internal amsmath macro which breaks to be prepended with
	% the \resetparens command
	\makeatletter
		\preto{\resetMathstrut@}{\resetparens}
		\makeatother

\usepackage[bookmarks,pagebackref=false]{hyperref}
	\usepackage[all]{hypcap}
	\hypersetup{
		,bookmarksopen  = false%
		,pdfborder      = false%
		,colorlinks     = false%
	%	,colorlinks     = true%
	%	,urlcolor       = blue%
	%	,linkcolor      = black%
	%	,citecolor      = black%
	%	,filecolor      = black%
		,setpagesize    = false%
		,pdfstartview   = {XYZ null null 1}%
		,pdfprintscaling= None%
		,pdfpagelayout  = TwoPageLeft%
		,verbose        = true%
	}
	\hypersetup{ pdfauthor = {Justin Willmert} }%

\begin{document}

\cardfrontstyle{headings}

%%%%%%%%%%%%%%%%%%%%%%%%%%%%%%%%%%%%%%%%%%%%%%%%%%%%%%%%%%%%%%%%%%%%%%%%%%%%%%%
%%%% Electrodynamics
%%%%%%%%%%%%%%%%%%%%%%%%%%%%%%%%%%%%%%%%%%%%%%%%%%%%%%%%%%%%%%%%%%%%%%%%%%%%%%%
\cardfrontfoot{Electrodynamics}

\begin{flashcard}{Maxwell's Equations (Guassian)}
	\begin{align*}
		\vec ∇ · \vec D &= 4πρ_f
			& \vec ∇ × \vec E &= -\frac 1c \frac{∂\vec B}{∂t}
		\\
		\vec ∇ · \vec B &= 0
			& \vec ∇ × \vec H &= \frac 1c ( \frac{∂\vec D}{∂t} + 4π \vec{J_f} )
	\end{align*}
\end{flashcard}

\begin{flashcard}{Maxwell's Equations (SI)}
	\begin{align*}
		\vec ∇ · \vec D &= ρ_f
			& \vec ∇ × \vec E &= -\frac{∂\vec B}{∂t}
		\\
		\vec ∇ · \vec B &= 0
			& \vec ∇ × \vec H &= \frac{∂\vec D}{∂t} + \vec{J_f}
	\end{align*}
\end{flashcard}

%%%%%%%%%%%%%%%%%%%%%%%%%%%%%%%%%%%%%%%%%%%%%%%%%%%%%%%%%%%%%%%%%%%%%%%%%%%%%%%
%%%% General Math
%%%%%%%%%%%%%%%%%%%%%%%%%%%%%%%%%%%%%%%%%%%%%%%%%%%%%%%%%%%%%%%%%%%%%%%%%%%%%%%
\cardfrontfoot{General Math}

\begin{flashcard}{Gaussian Integrals}
	\begin{align*}
		I_n(x) = \!\! ∫_0^∞ \!\! x^n e^{-ax²} dx &=
			\begin{cases}
				\displaystyle
				\frac{1}{2} \sqrt{\frac{π}{a^{m+1}}} \frac{(2m)!}{4^m m!}
					& \text{$n = 2m$} \\
				\displaystyle
				\frac{1}{2} \frac{1}{a^{k+1}} k!
					& \text{$n = 2k + 1$}
			\end{cases}
	\end{align*}
	\vspace{-\baselineskip}
	\begin{align*}
		I₀(x) &= \frac 12 \sqrt{\frac π a}
			& I₁(x) &= \frac 1 {2a} \\
		I₂(x) &= \frac{1}{4a} \sqrt{\frac π a}
			& I₃(x) &= \frac 1 {2a²}
	\end{align*}
\end{flashcard}

\begin{flashcard}{Geometric Series}
	\begin{align*}
		\sum_{i=0}^N r^i &= \frac{1 - r^{N+1}}{1 - r} \\
		\sum_{i=0}^∞ r^i &= \frac{1}{1 - r}
	\end{align*}
\end{flashcard}

\begin{flashcard}{Stirling's Approximation}
	\[ n! ≈ (\frac ne)^n \sqrt{2πn} \] \\
	\[ \ln n! ≈ n\ln n - n \]
\end{flashcard}

%%%%%%%%%%%%%%%%%%%%%%%%%%%%%%%%%%%%%%%%%%%%%%%%%%%%%%%%%%%%%%%%%%%%%%%%%%%%%%%
%%%% MECHANICS
%%%%%%%%%%%%%%%%%%%%%%%%%%%%%%%%%%%%%%%%%%%%%%%%%%%%%%%%%%%%%%%%%%%%%%%%%%%%%%%
\cardfrontfoot{Mechanics}

\begin{flashcard}{Bernoulli's equation}
	\[ \frac{v²}{2} + gz + \frac{p}{ρ} = \text{constant} \]
\end{flashcard}

%%%%%%%%%%%%%%%%%%%%%%%%%%%%%%%%%%%%%%%%%%%%%%%%%%%%%%%%%%%%%%%%%%%%%%%%%%%%%%%
%%%% THERMODYNAMICS
%%%%%%%%%%%%%%%%%%%%%%%%%%%%%%%%%%%%%%%%%%%%%%%%%%%%%%%%%%%%%%%%%%%%%%%%%%%%%%%
\cardfrontfoot{Thermodynamics}

\begin{flashcard}{Adiabatic Process}
	Also called \emph{isentropic}. $ΔS = 0$ in the process. Use the
	thermodynamic identity at constant volume and a systems internal energy
	equation to derive properties about the entropy of the system.
\end{flashcard}

\begin{flashcard}{Adiabatic Properties of Ideal Gas}
	\begin{align*}
		T₁ {V₁}^{γ-1} &= \text{const} \\
		{T₁}^{γ/(1-γ)} P₁ &= \text{const} \\
		P₁ {V₁}^γ &= \text{const}
	\end{align*}
\end{flashcard}

\begin{flashcard}{Bose-Einstein Distribution}
	\[ f(ε) = \frac{1}{e^{(ε-μ)/k_B T} - 1} \]
\end{flashcard}

\begin{flashcard}{Carnot Efficiency}
	\[ η = 1 - \frac{T_l}{T_h} \]
\end{flashcard}

\begin{flashcard}{Carnot Cycle}
	Characterized by alternating stages of isothermal and isentropic expansion
	and compression. Work done is
	\[ W = (T_h - T_l)(S_H - S_L) \]
	where $T_l$ and $T_h$ are the low and high temperatures reached during the
	cycle and $S_L$ and $S_H$ are the low and high entropies of the working
	substance.
\end{flashcard}

\begin{flashcard}{Equipartition Theorem}
	A classical gas's energy gains $\frac 12 k_B T$ for each degree of
	freedom. An ideal monotomic gas has
		$ U = \frac 32 k_B T $
	from three translational degrees of freedom, while an ideal diatomic gas
	has
		$ U = \frac 52 k_B T $
	from an additional two degrees of rotational freedom.
\end{flashcard}

\begin{flashcard}{Fermi Gasses}
	\begin{enumerate}
		\item High kinetic energy
		\item Low heat capacity
		\item Low magnetic susceptibility
		\item Low interparticle collision rate
		\item High pressure
	\end{enumerate}
\end{flashcard}

\begin{flashcard}{Fermi-Dirac Distribution}
	\[ f(ε) = \frac{1}{e^{(ε-μ)/k_B T} + 1} \]
\end{flashcard}

\begin{flashcard}{Gibbs Free Energy}
	\[ G ≡ U + PV - TS \]
\end{flashcard}

\begin{flashcard}{Helmholtz Free Energy}
	Acts as effective energy in isothermal changes of volume.
	\[ F ≡ U - TS \]
	\[ dF = dU - S dT \]
\end{flashcard}

\begin{flashcard}{Ideal Gasses}
	\[ PV = nRT \]
	\[ PV = N k_B T \]
	\[ Z_N = \frac{Z₁^N}{N!} \]	
\end{flashcard}

\begin{flashcard}{Ideal Gas (RMS Average Speed)}
	Derived by considering a single particle. For translation in three
	dimensions $KE = \frac 32 k_B T$  and also $KE = \frac 12 mv²$ so that
	when combined,
	\begin{align*}
		\frac 12 mv² &= \frac 32 k_B T \\
		v &= \sqrt{\frac{3 k_B T}{m}}
	\end{align*}
\end{flashcard}

\begin{flashcard}{Ideal Monoatomic Gas}
	\begin{align*}
		C_V &= \frac 32 Nk_B
			& C_P &= \frac 52 Nk_B \\
		U &= \frac 32 Nk_B T
			& γ &= \frac 53
	\end{align*}
\end{flashcard}

\begin{flashcard}{Maxwell Speed Distribution}
	\[ f(v) = \sqrt{ (\frac{m}{2π k_B T})³ } 4πv² \exp(-\frac{mv²}{2k_B T}) \]
	\begin{align*}
		v_\mathrm{rms} &= \sqrt{\frac{3k_B T}{m}}
			& ⟨v⟩ &= \sqrt{\frac{8k_B T}{πm}}
	\end{align*}
\end{flashcard}

\begin{flashcard}{Partition Function}
	\[ Z = \sum_n e^{-ε_n / k_B T} \]
	\begin{align*}
		U &= k_B T² \frac{∂ \ln Z}{∂T}
			& F = -k_B T \ln Z
	\end{align*}
\end{flashcard}

\begin{flashcard}{Photon Gasses}
	\[ U = σ_b VT⁴ \]
	\[ P = \frac 13 σ_b VT⁴ \]
	\[ μ = 0 \]
\end{flashcard}

\begin{flashcard}{Planck Distribution function}
	\[ ⟨s⟩ = \frac{1}{e^{ℏω/k_B T} - 1} \]
\end{flashcard}

\begin{flashcard}{Planck Spectral Density (frequency)}
	\[ u_ω = \frac{ℏ}{π²c³} \frac{ω³}{e^{ℏω/k_B T} - 1} \]
\end{flashcard}

\begin{flashcard}{Radiant Energy Flux (blackbody)}
	\[ J_u = \frac{π²{k_B}⁴}{60ℏ³c²} T⁴ \]
	\[ J_u = \frac{c}{4} u \]
\end{flashcard}

\begin{flashcard}{Stefan-Boltzmann Law (energy density)}
	\begin{align*}
		\frac{U}{V} = u &= \frac{π²{k_B}³}{15ℏ³c³} T⁴ \\
			u &= σ_B T⁴
	\end{align*}
	\[ u = \frac{4}{c} J_u \]
\end{flashcard}

\begin{flashcard}{Thermodynamic Identity}
	\[ dU = T dS - P dV + μ dN \]
	\begin{align*}
		C_V &= (\frac{∂U}{∂T})_V = T (\frac{∂S}{∂T})_V
			& P = -(\frac{∂U}{∂V})_S
	\end{align*}
\end{flashcard}

\end{document}
