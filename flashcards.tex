\documentclass[avery5371]{flashcards}
\usepackage{defaults-utilities}
\usepackage{defaults-fonts}
\usepackage{defaults-math}
\usepackage{defaults-layout}

\title{University of Minnesota Physics GWE Flashcards}
\author{Justin Willmert}

\begin{document}

\cardfrontstyle{headings}

%%%%%%%%%%%%%%%%%%%%%%%%%%%%%%%%%%%%%%%%%%%%%%%%%%%%%%%%%%%%%%%%%%%%%%%%%%%%%%%
%%%% Electrodynamics
%%%%%%%%%%%%%%%%%%%%%%%%%%%%%%%%%%%%%%%%%%%%%%%%%%%%%%%%%%%%%%%%%%%%%%%%%%%%%%%
\cardfrontfoot{Electrodynamics}

\begin{flashcard}{Electrical component voltages}
	\begin{align*}
		V &= IR		&	V = \frac{Q}{C} \\
		V &= L\frac{dI}{dt}
	\end{align*}
\end{flashcard}

\begin{flashcard}{Maxwell's Equations (Guassian)}
	\begin{align*}
		\vec ∇ · \vec D &= 4πρ_f
			& \vec ∇ × \vec E &= -\frac 1c \frac{∂\vec B}{∂t}
		\\
		\vec ∇ · \vec B &= 0
			& \vec ∇ × \vec H &= \frac 1c ( \frac{∂\vec D}{∂t} + 4π \vec{J_f} )
	\end{align*}
\end{flashcard}

\begin{flashcard}{Maxwell's Equations (SI)}
	\begin{align*}
		\vec ∇ · \vec D &= ρ_f
			& \vec ∇ × \vec E &= -\frac{∂\vec B}{∂t}
		\\
		\vec ∇ · \vec B &= 0
			& \vec ∇ × \vec H &= \frac{∂\vec D}{∂t} + \vec{J_f}
	\end{align*}
\end{flashcard}

%%%%%%%%%%%%%%%%%%%%%%%%%%%%%%%%%%%%%%%%%%%%%%%%%%%%%%%%%%%%%%%%%%%%%%%%%%%%%%%
%%%% General Math
%%%%%%%%%%%%%%%%%%%%%%%%%%%%%%%%%%%%%%%%%%%%%%%%%%%%%%%%%%%%%%%%%%%%%%%%%%%%%%%
\cardfrontfoot{General Math}

\begin{flashcard}{Gaussian Integrals}
	\begin{align*}
		I_n(x) = \!\! ∫_0^∞ \!\! x^n e^{-ax²} dx &=
			\begin{cases}
				\displaystyle
				\frac{1}{2} \sqrt{\frac{π}{a^{m+1}}} \frac{(2m)!}{4^m m!}
					& \text{$n = 2m$} \\
				\displaystyle
				\frac{1}{2} \frac{1}{a^{k+1}} k!
					& \text{$n = 2k + 1$}
			\end{cases}
	\end{align*}
	\vspace{-\baselineskip}
	\begin{align*}
		I₀(x) &= \frac 12 \sqrt{\frac π a}
			& I₁(x) &= \frac 1 {2a} \\
		I₂(x) &= \frac{1}{4a} \sqrt{\frac π a}
			& I₃(x) &= \frac 1 {2a²}
	\end{align*}
\end{flashcard}

\begin{flashcard}{Geometric Series}
	\begin{align*}
		\sum_{i=0}^N r^i &= \frac{1 - r^{N+1}}{1 - r} \\
		\sum_{i=0}^∞ r^i &= \frac{1}{1 - r}
	\end{align*}
\end{flashcard}

\begin{flashcard}{Stirling's Approximation}
	\[ n! ≈ (\frac ne)^n \sqrt{2πn} \] \\
	\[ \ln n! ≈ n\ln n - n \]
\end{flashcard}

%%%%%%%%%%%%%%%%%%%%%%%%%%%%%%%%%%%%%%%%%%%%%%%%%%%%%%%%%%%%%%%%%%%%%%%%%%%%%%%
%%%% MECHANICS
%%%%%%%%%%%%%%%%%%%%%%%%%%%%%%%%%%%%%%%%%%%%%%%%%%%%%%%%%%%%%%%%%%%%%%%%%%%%%%%
\cardfrontfoot{Mechanics}

\begin{flashcard}{Bernoulli's equation}
	\[ \frac{v²}{2} + gz + \frac{p}{ρ} = \text{constant} \]
\end{flashcard}

%%%%%%%%%%%%%%%%%%%%%%%%%%%%%%%%%%%%%%%%%%%%%%%%%%%%%%%%%%%%%%%%%%%%%%%%%%%%%%%
%%%% THERMODYNAMICS
%%%%%%%%%%%%%%%%%%%%%%%%%%%%%%%%%%%%%%%%%%%%%%%%%%%%%%%%%%%%%%%%%%%%%%%%%%%%%%%
\cardfrontfoot{Thermodynamics}

\begin{flashcard}{Adiabatic Process}
	Also called \emph{isentropic}. $ΔS = 0$ in the process. Use the
	thermodynamic identity at constant volume and a systems internal energy
	equation to derive properties about the entropy of the system.
\end{flashcard}

\begin{flashcard}{Adiabatic Properties of Ideal Gas}
	\begin{align*}
		T₁ {V₁}^{γ-1} &= \text{const} \\
		{T₁}^{γ/(1-γ)} P₁ &= \text{const} \\
		P₁ {V₁}^γ &= \text{const}
	\end{align*}
\end{flashcard}

\begin{flashcard}{Bose-Einstein Distribution}
	\[ f(ε) = \frac{1}{e^{(ε-μ)/k_B T} - 1} \]
\end{flashcard}

\begin{flashcard}{Carnot Efficiency}
	\[ η = 1 - \frac{T_l}{T_h} \]
\end{flashcard}

\begin{flashcard}{Carnot Cycle}
	Characterized by alternating stages of isothermal and isentropic expansion
	and compression. Work done is
	\[ W = (T_h - T_l)(S_H - S_L) \]
	where $T_l$ and $T_h$ are the low and high temperatures reached during the
	cycle and $S_L$ and $S_H$ are the low and high entropies of the working
	substance.
\end{flashcard}

\begin{flashcard}{Equipartition Theorem}
	A classical gas's energy gains $\frac 12 k_B T$ for each degree of
	freedom. An ideal monotomic gas has
		$ U = \frac 32 k_B T $
	from three translational degrees of freedom, while an ideal diatomic gas
	has
		$ U = \frac 52 k_B T $
	from an additional two degrees of rotational freedom.
\end{flashcard}

\begin{flashcard}{Fermi Gasses}
	\begin{enumerate}
		\item High kinetic energy
		\item Low heat capacity
		\item Low magnetic susceptibility
		\item Low interparticle collision rate
		\item High pressure
	\end{enumerate}
\end{flashcard}

\begin{flashcard}{Fermi-Dirac Distribution}
	\[ f(ε) = \frac{1}{e^{(ε-μ)/k_B T} + 1} \]
\end{flashcard}

\begin{flashcard}{Gibbs Free Energy}
	\[ G ≡ U + PV - TS \]
\end{flashcard}

\begin{flashcard}{Helmholtz Free Energy}
	Acts as effective energy in isothermal changes of volume.
	\[ F ≡ U - TS \]
	\[ dF = dU - S dT \]
\end{flashcard}

\begin{flashcard}{Ideal Gasses}
	\[ PV = nRT \]
	\[ PV = N k_B T \]
	\[ Z_N = \frac{Z₁^N}{N!} \]	
\end{flashcard}

\begin{flashcard}{Ideal Gas (RMS Average Speed)}
	Derived by considering a single particle. For translation in three
	dimensions $KE = \frac 32 k_B T$  and also $KE = \frac 12 mv²$ so that
	when combined,
	\begin{align*}
		\frac 12 mv² &= \frac 32 k_B T \\
		v &= \sqrt{\frac{3 k_B T}{m}}
	\end{align*}
\end{flashcard}

\begin{flashcard}{Ideal Monoatomic Gas}
	\begin{align*}
		C_V &= \frac 32 Nk_B
			& C_P &= \frac 52 Nk_B \\
		U &= \frac 32 Nk_B T
			& γ &= \frac 53
	\end{align*}
\end{flashcard}

\begin{flashcard}{Maxwell Speed Distribution}
	\[ f(v) = \sqrt{ (\frac{m}{2π k_B T})³ } 4πv² \exp(-\frac{mv²}{2k_B T}) \]
	\begin{align*}
		v_\mathrm{rms} &= \sqrt{\frac{3k_B T}{m}}
			& ⟨v⟩ &= \sqrt{\frac{8k_B T}{πm}}
	\end{align*}
\end{flashcard}

\begin{flashcard}{Partition Function}
	\[ Z = \sum_n e^{-ε_n / k_B T} \]
	\begin{align*}
		U &= k_B T² \frac{∂ \ln Z}{∂T}
			& F = -k_B T \ln Z
	\end{align*}
\end{flashcard}

\begin{flashcard}{Photon Gasses}
	\[ U = σ_b VT⁴ \]
	\[ P = \frac 13 σ_b VT⁴ \]
	\[ μ = 0 \]
\end{flashcard}

\begin{flashcard}{Planck Distribution function}
	\[ ⟨s⟩ = \frac{1}{e^{ℏω/k_B T} - 1} \]
\end{flashcard}

\begin{flashcard}{Planck Spectral Density (frequency)}
	\[ u_ω = \frac{ℏ}{π²c³} \frac{ω³}{e^{ℏω/k_B T} - 1} \]
\end{flashcard}

\begin{flashcard}{Radiant Energy Flux (blackbody)}
	\[ J_u = \frac{π²{k_B}⁴}{60ℏ³c²} T⁴ \]
	\[ J_u = \frac{c}{4} u \]
\end{flashcard}

\begin{flashcard}{Stefan-Boltzmann Law (energy density)}
	\begin{align*}
		\frac{U}{V} = u &= \frac{π²{k_B}³}{15ℏ³c³} T⁴ \\
			u &= σ_B T⁴
	\end{align*}
	\[ u = \frac{4}{c} J_u \]
\end{flashcard}

\begin{flashcard}{Thermodynamic Identity}
	\[ dU = T dS - P dV + μ dN \]
	\begin{align*}
		C_V &= (\frac{∂U}{∂T})_V = T (\frac{∂S}{∂T})_V
			& P = -(\frac{∂U}{∂V})_S
	\end{align*}
\end{flashcard}

\end{document}
