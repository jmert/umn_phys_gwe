%%%%%%%%%%%%%%%%%%%%%%%%%%%%%%%%%%%%%%%%%%%%%%%%%%%%%%%%%%%%%%%%%%%%%%%%%%%%%%%
%%%% Problem 8
%%%%%%%%%%%%%%%%%%%%%%%%%%%%%%%%%%%%%%%%%%%%%%%%%%%%%%%%%%%%%%%%%%%%%%%%%%%%%%%
\problem{8}
\subsubsection{Question}
% Keywords
	\index{thermodynamics!Nitrogen velocity}

Estimate (a) the average speed (in \si{\m\per\s}) and (b) the mean free path
(in \si{\m}) of a nitrogen molecule in this room.

\subsubsection{Answer}

\begin{enumerate}[(a)]
	\item
		We relate the kinetic energy of an $N_2$ molecule with the thermal
		energy by the equipartition theorem. Since there are 3 translational
		degrees of freedom,
		\begin{align*}
			\frac 12 mv² &= \frac 32 k_B T \\
			v &= \sqrt{\frac{3 k_B T}{m}}
		\end{align*}
		The mass of the molecule is twice that of a single nitrogen atom
		which is itself about 14 proton masses. Therefore
		\begin{empheq}[box=\fbox]{align}
			v &≈ \sqrt{\frac{3 k_B T}{28 m_p}} \\
			v &≈ \SI{515}{\m\per\s}
		\end{empheq}
	\item
		Two particles collide if they come within $2r₀$ of each other where
		$r₀$ is the typical radius of the particle. For diatomic nitrogen,
		we assume $r₀ ≈ 2a₀$ where $a₀$ is the Bohr radius. Then in the time
		$τ$ that the particle is moving at velocity $⟨v⟩$, the particle can
		collide with any other particle within the swept-out volume
		\begin{align*}
			\mathcal V &= π(2r₀)² ⋅ ⟨v⟩τ
		\end{align*}
		Since there are $n$ particles per unit volume, there are $\mathcal N$
		atoms to collide with:
		\begin{align*}
			\mathcal N &= n\mathcal V = 4πn{r₀}² ⟨v⟩τ
		\end{align*}
		On average then, there are $\mathcal N$ collisions per length $⟨v⟩τ$
		traversed, or in its reciprocal form, the mean free path $λ$ is
		\begin{align*}
			λ &= \frac{1}{4πn{r₀}²}
		\end{align*}
		To estimate the particle density, consider the ideal gas law
		$PV=Nk_BT$. We can assume atmospheric pressure at room temperature,
		so the density is
		\begin{align*}
			n &= \frac N V = \frac{P}{k_B T}
		\end{align*}
		Putting it all together,
		\begin{empheq}[box=\fbox]{align}
			λ &≈ \frac{k_B T}{4π{r₀}²P} \\
			{}&≈ \SI{2.90e-7}{\m}
		\end{empheq}
\end{enumerate}
