%%%%%%%%%%%%%%%%%%%%%%%%%%%%%%%%%%%%%%%%%%%%%%%%%%%%%%%%%%%%%%%%%%%%%%%%%%%%%%%
%%%% Problem 2
%%%%%%%%%%%%%%%%%%%%%%%%%%%%%%%%%%%%%%%%%%%%%%%%%%%%%%%%%%%%%%%%%%%%%%%%%%%%%%%
\problem{2}
\subsubsection{Question}
% Keywords
	\index{mechanics!Impulse on a rod}

If an impulse is delivered to the end of a uniform rod of length $ℓ$, lying on
a frictionless plane, how far will it travel while making one revolution? The
impulse is in the plane of the table and perpendicular to the rod.

\subsubsection{Answer}

For a given impulse $\vec J$, the change in the motion is $\vec J = Δ\vec p$.
If the rod start at rest, then the final momentum must be $\vec p = \vec J$.
This means the rod is moving laterally with a velocity
\begin{align*}
    V = \frac 1m \vec J
\end{align*}
which when integrated over a time $t$ gives the distance it has moved $\vec x$.
\begin{align*}
    \vec x = \frac 1m \vec J t
\end{align*}

The impulse also imparts a rotation on the rod because the force was not
applied at the rod's center of mass. The torque $\vec τ$ relates the force
to the angular momentum $\vec L$ by
\begin{align*}
    \vec r × \vec F &= \vec τ = \dot{\vec L}
\end{align*}
Integrating both sides of the equation, we can write the equation in terms of
the given impulse:
\begin{align*}
    \vec r × \int \vec F \,dt &= \int \dot{\vec L} \,dt \\
    \vec r × \vec J &= Δ\vec L
\end{align*}
Again, since the rod starts at rest, we know that the final angular momentum
must be
\begin{align*}
    \vec L = \vec r × \vec J
\end{align*}
The rotation about the rod's center of mass  occurs at a rate $\vec ω$
dependent on the moment of inertia $I = \frac{1}{12} mℓ²$, so
\begin{align*}
    \vec ω = \frac{12}{mℓ²} \vec r × \vec J
\end{align*}
We know that the impulse is applied perpendicular to the rod, so we can easily
integrate the expression in time and solve for the time it takes to revolve
$2π$ radians:
\begin{align*}
    θ = 2π &= \frac{12}{mℓ²} rJt \\
    t &= \frac{πmℓ²}{6 rJ}
\end{align*}

Plugging this back into the linear motion equation, the rod travels
\begin{align*}
    \vec x = \frac{1}{m} \vec J ⋅ \frac{πmℓ²}{6 rJ}
\end{align*}
where we can set $r = \frac 12 ℓ$ and therefore simplifies to
\begin{align}
    \boxed{
    \vec x = \frac{πℓ}{3} \hat J
    }
\end{align}
where $\hat J$ is the direction of the applied impulse.

%%%%%%%%%%%%%%%%%%%%%%%%%%%%%%%%%%%%%%%%%%%%%%%%%%%%%%%%%%%%%%%%%%%%%%%%%%%%%%%
%%%% Problem 3
%%%%%%%%%%%%%%%%%%%%%%%%%%%%%%%%%%%%%%%%%%%%%%%%%%%%%%%%%%%%%%%%%%%%%%%%%%%%%%%
\problem{3}
\subsubsection{Question}
% Keywords
	\index{electrodynamics!Properties of a magnetic field}

A time-indpendent magnetic field is given by $\vec B = 2bxy \,\hat ı +
ay² \,\hat ȷ$.
\begin{enumerate}[a)]
    \item
        What is the relationship between the constants $a$ and $b$?
    \item
        Determine the steady current density $J$ that gives rise to this field.
\end{enumerate}

\subsubsection{Answer}
For part (a), we realize that all magnetic fields must be divergenceless.
Therefore we can find the requirements on the constants $a$ and $b$ by
constraining the divergence to be zero.
\begin{align*}
    \vec ∇ ⋅ \vec B = 0 &= \frac{∂}{∂x}(2bxy) + \frac{∂}{∂y}(ay²) \\
    0 &= 2by + 2ay \\
    b &= -a
\end{align*}
Therefore the relation between the constants is that
\begin{align}
    \boxed {b = -a}
\end{align}

For the second part, we make use of Maxwell's equations. Assuming that none of
the field is due to a time-varying electric field, we make use of
\begin{align*}
    \vec ∇ × \vec B &= μ₀ \vec J
\end{align*}
to calculate the current that generates the field. Doing so, we find that the
solution is
\begin{align}
    \boxed{ \vec J = \frac{2a}{μ₀} x \,\hat k }
\end{align}

%%%%%%%%%%%%%%%%%%%%%%%%%%%%%%%%%%%%%%%%%%%%%%%%%%%%%%%%%%%%%%%%%%%%%%%%%%%%%%%
%%%% Problem 4
%%%%%%%%%%%%%%%%%%%%%%%%%%%%%%%%%%%%%%%%%%%%%%%%%%%%%%%%%%%%%%%%%%%%%%%%%%%%%%%
\problem{4}
\subsubsection{Question}
% Keywords
	\index{electrodynamics!Charges from multipole moments}

A set of four point charges $q₁$, $q₂$, $q₃$, and $q₄$ are arranged
collinearly along the $z$-axis at $z₁ = 0$, $z₂ = a$, $z₃ = 2a$, $z₄ = 4a$,
respectively and the resulting electric field at a distant point $\vec r$ ($r
≫ a$) decays \emph{faster} than $1/r³$. Determine the values of $q₁$ and $q₄$
which $q₂ = +2$ and $q₃ = +4$. Units for all charges are Coulombs.

\subsubsection{Answer}

Given that the electric field must fall off faster than $1/r³$, this
corresponds to a potential which drops off faster than $1/r²$. We know that
the monopole moment drops off like $1/r$ and the dipole like $1/r²$, so we
conclude that the first configuration which could satisfy the given
requirement is that of a quadrupole moment.

Making use of the fact that he monopole and dipole moments are vanishing, we
can use them to generate constraint equations for what the charges must be:
we have two unknown charges and the two equations will allow us to solve them.

For the monopole, the sum of all charges must simply equal zero. Therefore
we immediately know that
\begin{align*}
    0 &= q₁ + q₄ + 6 \\
    -6 &= q₁ + q₄
\end{align*}

The dipole moment (where we take the dipole considered at the origin) is given
by
\begin{align*}
    \vec p = \sum_i \vec{r_i} q_i
\end{align*}
This gives us the equation
\begin{align*}
    0 &= 10a + 4aq₄ \\
    q₄ &= -\frac 52
\end{align*}
The charge $q₁$ does not show up in the equation since it is located at the
origin. This lets us very simply then solve for $q₁$ as
\begin{align*}
    -6 &= q₁ - \frac 52
\end{align*}
Therefore, the solution is that the charges have values of
\begin{align}
    \boxed{ q₁ = -\frac 72 } \\
    \boxed{ q₄ = -\frac 52 }
\end{align}

%%%%%%%%%%%%%%%%%%%%%%%%%%%%%%%%%%%%%%%%%%%%%%%%%%%%%%%%%%%%%%%%%%%%%%%%%%%%%%%
%%%% Problem 5
%%%%%%%%%%%%%%%%%%%%%%%%%%%%%%%%%%%%%%%%%%%%%%%%%%%%%%%%%%%%%%%%%%%%%%%%%%%%%%%
\problem{5}
\subsubsection{Question}
% Keywords
	\index{quantum mechanics!Spectral emission line width}

The Lyman-α transition in atomic hydrogen has a wavelength $λ =
\SI{121.5}{\nm}$, and a transition rate of \SI{0.6e9}{\s^{-1}}. Estimate the
minimum value of $Δλ/λ$.

\subsubsection{Answer}

We can make an estimate of the spread $Δλ$ by making use of the Heisenberg
uncertainty relation for energy-time. Starting with the variation in
wavelength,
\begin{align*}
    Δλ &= λ - λ' \\
    {} &= \frac{hc}{E} - \frac{hc}{E'} \\
    {} &= \frac{hc(E' - E)}{E E'}
\intertext{Making use of the approximation that $E ≈ E'$,}
    {} &= \frac{hcΔE}{E²}
\end{align*}
Dividing by the frequency and substituting in the uncertainty relation $ΔEΔt =
\frac{ℏ}{2}$,
\begin{align*}
    \frac{Δλ}{λ} &= \frac{hc}{λ} ⋅ \frac{1}{E²}\frac{ℏ}{2Δt} \\
    {} &= \frac{λ}{4πcΔt}
\end{align*}
For the time, we estimate the transition rate is occuring as fast as it can
within the limits of the uncertainty relation, so we can let $Δt ≈ \SI{0.6e9}
{\s^{-1}}$. Plugging in the other values, we find the fractional line width
to be estimated as
\begin{align}
    \boxed{ \frac{Δλ}{λ} ≈ \num{1.935e-8} ≈ \text{1 part in 50 million} }
\end{align}

%%%%%%%%%%%%%%%%%%%%%%%%%%%%%%%%%%%%%%%%%%%%%%%%%%%%%%%%%%%%%%%%%%%%%%%%%%%%%%%
%%%% Problem 11
%%%%%%%%%%%%%%%%%%%%%%%%%%%%%%%%%%%%%%%%%%%%%%%%%%%%%%%%%%%%%%%%%%%%%%%%%%%%%%%
\problem{11}
\subsubsection{Question}
% Keywords
	\index{statistical mechanics!Radiometric dating from mass ratios}

A rock is found to contain \SI{4.20}{\mg} of ${}^{238}U$ and \SI{2.00}{\mg}
of ${}^{206}Pb$. Assume tha the rock contained no lead at the time of its
formation, so that all the lead now present is due to th decay of the
uranium orignally present in the rock. Find the age of the rock given that
the half-life of ${}^{238}U$ is \SI{4.47e9}{\year}. The decay times of all
intermediate elements are negligibly short and ignore any differences in the
binding energies.

\subsubsection{Answer}

From decay processes, we know that the uranium atom count will decrease as an
exponential according to
\begin{align*}
    N_U = N_{U0}e^{-t/τ}
\end{align*}
where $τ = t_{1/2}/\ln 2$. Likewise, the number of lead atoms will increase
according to
\begin{align*}
    N_{Pb} = N_{U0} (1 - e^{-t/τ})
\end{align*}
Solveing for $N_{U0}$ in the first equation and substituting it into the
second, we can solve for the time required to generate a specific number of
uranium and lead atoms in a sample.
\begin{align*}
    N_{Pb} &= N_U e^{t/τ} (1 - e^{-t/τ}) \\
    t &= τ \ln(\frac{N_{Pb}}{N_U} + 1) \\
    t &= \frac{t_{1/2}}{\ln 2} \ln(\frac{N_{Pb}}{N_U} + 1)
\end{align*}
We were only given the masses, though, so we approximate the mass of each
atom by the number of nucleons in the nucleus; each uranium atom has a mass
of $m_U = 238m_N$ making the $N_U$ atoms have a mass of $M_U = 238 N_U m_N$,
and similar for the lead. This gives us the final equation
\begin{align*}
    t &= \frac{t_{1/2}}{\ln 2} \ln(\frac{238}{206} \frac{M_{Pb}}{M_U} + 1)
\end{align*}
Plugging in all the numbers,
\begin{align}
    \boxed{ t = \SI{2.83e9}{\year} }
\end{align}
