%%%%%%%%%%%%%%%%%%%%%%%%%%%%%%%%%%%%%%%%%%%%%%%%%%%%%%%%%%%%%%%%%%%%%%%%%%%%%%%
%%%% Problem 1
%%%%%%%%%%%%%%%%%%%%%%%%%%%%%%%%%%%%%%%%%%%%%%%%%%%%%%%%%%%%%%%%%%%%%%%%%%%%%%%
\problem{1}
\subsubsection{Question}
% Keywords
	\index{quantum!Expectation values}

An electron in a hydrogen atom occupies a state:
\begin{align*}
	\ket{ψ} &= \sqrt{\frac 13}\ket{3,1,0,+} + \sqrt{\frac 23} \ket{2,1,1,-}
\end{align*}
where the properly normalized states are specified by the quantum numbers 
$\ket{n,ℓ,m,±}$ and the $±$ specifies whether the spin is up or down.

\makeatletter
\newcommand{\interitemtext}[1]{%
	\begin{list}{}
		{
			\itemindent=0mm\labelsep=0mm
			\labelwidth=0mm\leftmargin=0mm
			\addtolength{\leftmargin}{-\@totalleftmargin}
		}
		\item #1
	\end{list}
}
\makeatother

\renewcommand{\labelenumi}{(\alph{enumi})}
\begin{enumerate}
	\item
		What is the expectation value of the energy in terms of the ground
		state energy?
	\item
		If you meausred the expectation values of the orbital momentum squared
		$\expect{L²}$, the square of the spin $\expect{S²}$, and their
		$z$-components $\expect{L_z}$ and $\expect{S_z}$, what would be the
		result?
	\item
		Show that if you measure the position of the electron, the probability
		density for finding it an an angle specified by $θ$ and $ϕ$
		integrated over all values of $r$ is independent of $θ$ and $ϕ$. Note,
		for this part you will need $Y_1^0 = \sqrt{3/4π}\cos θ$ and $Y_1^1 =
		-\sqrt{3/8π} \sin θ \exp(iϕ)$. You do \emph{not}, however, need to
		know the radial functions, only that they are properly normalized and
		orthogonal to each other.
	\item
		List all additional possible states that are degenerate with the first
		state in the linear combination above. Note: this part can be done
		even if you have not answered the previous parts.
\interitemtext{Assume now that the state $\ket{ψ}$, given above, is the 
initial state of an electron in a hydrogen atom.}
	\item
		Write down the elctron's state as a function of time for all $t>0$.
	\item
		go through the results you obtained in parts (a) through (c) and
		determine which of them are time independent.
\end{enumerate}

\subsubsection{Answer}
\begin{enumerate}
	\item
		Calculate the energy by sandwhiching the Hamiltonian between the
		wavefunction:
		\begin{align*}
			\expect{E} &= \braopket{ψ}{H}{ψ} \\
			{} &= (\sqrt{\frac 13}\bra{3,1,0,+} + \sqrt{\frac 23}
				\bra{2,1,1,-})H(\sqrt{\frac 13}\ket{3,1,0,+} + \sqrt{\frac 23}
				\ket{2,1,1,-}) \\
			\begin{split}
			{} &= \frac 13 \braopket{3,1,0,+}{H}{3,1,0,+} + \frac{\sqrt{2}}{3}
				\braopket{2,1,1,-}{H}{3,1,0,+} \\
				&\quad + \frac{\sqrt{2}}{3} \braopket{3,1,0,+}{H}{2,1,1,-} + 
				\frac 23 \braopket{2,1,1,-}{H}{2,1,1,-}
			\end{split}\\
		\intertext{For every term, the wavefunctions are eigenstates of the
		Hamiltonian, so we extract the appropriate energy term from every
		bra-ket sandwhich. Then the middle two terms integrate to zero since
		states with different $n$ are orthogonal while the first and last terms
		integrate to unity since they are properly normalized.}
			\expect{E} &= \frac 13 E₃ + 0 + 0 + \frac 23 E₂ \\
		\intertext{Each energy is related to the ground state energy by
		$E_n = E₀/n²$, so}
			{} &= \frac 13 \frac{E₀}{9} + \frac 23 \frac{E₀}{4}
		\end{align*}
		\begin{empheq}[box=\fbox]{align}
			\expect{E} &= \frac{11}{54} E₀ ≈ \SI{-2.77}{\eV}
		\end{empheq}
	\item
		For each of the other expectation values, the process is very
		similar with an appropriate change for eigenvalues; specifically,
		\begin{align*}
			L²\ket{n,ℓ,m,±} &= ℓ(ℓ+1)ℏ²\ket{n,ℓ,m,±} \\
			S²\ket{n,ℓ,m,±} &= \frac 12(\frac 12 + 1)ℏ²\ket{n,ℓ,m,±} \\
			L_z\ket{n,ℓ,m,±} &= ℓℏ\ket{n,ℓ,m,±} \\
			S_z\ket{n,ℓ,m,±} &= ±\frac 12 ℏ\ket{n,ℓ,m,±}
		\end{align*}
		The same restrictions that the middle terms integrate to zero because
		of orthogonality and the first and last terms integrate to unity still
		applies, so we can almost immediately conclude that
		\begin{empheq}[box=\fbox]{align}
			\expect{L²} &= 2ℏ² \\
			\expect{S²} &= \frac{3ℏ²}{4} \\
			\expect{L_z} &= \frac{2ℏ}{3} \\
			\expect{S_z} &= -\frac{ℏ}{6}
		\end{empheq}
	\item
		In the $\ket{r,θ,ϕ}$ basis,
		\begin{align*}
			\ket{3,1,0} &= R_{3,1}(r) Y_1^0(θ,ϕ) = R_{31}(r)
				\sqrt{\frac{3}{4π}}\cos θ \\
			\ket{2,1,1} &= R_{2,1}(r) Y_1^1(θ,ϕ) = -R_{21}(r)
				\sqrt{\frac{3}{8π}} e^{iϕ} \sin θ \\
		\end{align*}
		This means that the probability density is
		\begin{align*}
			\braket{ψ}{ψ} &= \frac 13 \braket{3,1,0}{3,1,0} +
				\frac{\sqrt{2}}{3} \braket{2,1,1}{3,1,0} +
				\frac{\sqrt{2}}{3} \braket{3,1,0}{2,1,1} +
				\frac 23 \braket{2,1,1}{2,1,1} \\
			{} &= \frac{1}{4π}\cos² θ R_{31}²(r) - \frac{1}{π} \sin θ \cos θ
				(e^{iϕ} + e^{-iϕ}) R_{21}(r) R_{31}(r) + \frac{1}{4π}
				\sin² θ R_{21}²(r)
		\end{align*}
		Integrating over $r$,
		\begin{align*}
			\begin{split}
				∫_0^∞ \braket{ψ}{ψ} \dd r &=
					∫_0^∞ \frac{1}{4π}\cos² θ R_{31}²(r) -
					\frac{1}{π} \sin θ\cos θ (e^{iϕ} + e^{-iϕ})
					R_{21}(r)R_{31}(r)\\
					&\quad + \frac{1}{4π} \sin² θ R_{21}²(r) \dd r
			\end{split}\\
		\intertext{Integrating over all $r$, we know that $R_{nℓ}R_{n'ℓ'}$
		are orthonormal, so again the first and last terms' $R$ integrates to
		unity and the middle term integrates to zero.}
			{} &= \frac{1}{4π}(\cos² θ + \sin² θ)
		\end{align*}
		Therefore we find that the probability density is constant in $θ$ and
		$ϕ$ when integrated over all $r$.
		\begin{empheq}[box=\fbox]{align}
			∫_0^∞ \braket{ψ}{ψ} \dd r &= \frac{1}{4π}
		\end{empheq}
	\item
		The states degenerate with the first term in $ψ$ are all combinations
		of allowed $ℓ$, $m$, and $±$: $n$ must remain at $n=3$ since it is the
		$n$ quantum number which determines the energy of the state. The
		angular momentum number $ℓ$ has to be in the range $[0, n-1]$, so there
		are at least 3 cases.
		\begin{align*}
			\ket{3,0,m,±} \\
			\ket{3,1,m,±} \\
			\ket{3,2,m,±}
		\end{align*}
		Then for each $ℓ$, the projection $m$ can take a range of values
		$m ∈ [-ℓ,ℓ]$ so using $\{...,-1,0,-1,...\}$ to denote a set of options,
		\begin{align*}
			\ket{3,0,m,±} &\rightarrow \ket{3,0,\{0\},±}
				& \text{2 states} \\
			\ket{3,1,m,±} &\rightarrow \ket{3,1,\{-1,0,1\},±}
				& \text{6 states} \\
			\ket{3,2,m,±} &\rightarrow \ket{3,2,\{-2,-1,0,1,2\},±}
				& \text{10 states}
		\end{align*}
		\begin{center}
			\framebox{In total, there are 18 degenerate states}
		\end{center}
	\item
		To get the time evolution, we simply use the fact that for each basis
		eigenstate, we can add the time evolution component
		\begin{align*}
			\exp (-\frac{iE_nt}{ℏ})
		\end{align*}
		to get (in terms of the ground state energy $E₀$)
		\begin{empheq}[box=\fbox]{align}
			\ket{ψ(t)} &= \sqrt{\frac 13}\ket{3,1,0,+} e^{-iE₀t/9ℏ} +
				\sqrt{\frac 23} \ket{2,1,1,-} e^{-iE₀t/4ℏ}
		\end{empheq}
	\item
		From Ehrenfest's Theorem, we can quickly find the answers to most of
		the question without worrying about the wavefunction. Ehrenfest's
		Theorem is
		\begin{align*}
			\frac{\dd}{\dd t}\expect{E} &= -\frac iℏ \expect{[Ω,H]} +
				\expect{\frac{∂Ω}{∂t}}
		\end{align*}
		None of the operators $L²$, $S²$, $L_z$, and $S_z$ are explicit in
		time, so the second term on the right can be dropped. Then because
		each of these operators commute with the Hamiltonian, the first term
		on the right is also dropped. Therefore, the expectation values are
		constant in time, so
		\begin{empheq}[box=\fbox]{align}
			\expect{L²} &\quad\text{Time independent} \\
			\expect{S²} &\quad\text{Time independent} \\
			\expect{L_z} &\quad\text{Time independent} \\
			\expect{S_z} &\quad\text{Time independent}
		\end{empheq}
		For the probabilty density, we return to the integral in part (c) and
		insert the appropriate exponential terms. The first and last terms'
		exponentials cancel each other out, leaving
		\begin{align*}
			\begin{split}
				∫_0^∞ \braket{ψ}{ψ} \dd r &=
					∫_0^∞ \frac{1}{4π}\cos² θ R_{31}²(r) -
					\frac{1}{π} \sin θ\cos θ (e^{iϕ} + e^{-iϕ})
					R_{21}(r)R_{31}(r) \\
					&\quad · \left[ \exp(\frac{i(E₂-E₃)t}{ℏ}) + 
					\exp(-\frac{i(E₂-E₃)t}{ℏ}) \right] \\
					&\quad + \frac{1}{4π} \sin² θ R_{21}²(r) \dd r
			\end{split}
		\end{align*}
		The integral is unaffected by the new time factors, though, so
		integrating over $r$, the middle term still goes to zero and we're
		left with the same result previously of $1/4π$, therefore
		\begin{empheq}[box=\fbox]{align}
			∫_0^∞ \braket{ψ}{ψ} \dd r &\quad\text{Time independent}
		\end{empheq}		
\end{enumerate}
